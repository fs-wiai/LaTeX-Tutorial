\section{Vert. Arbeiten}
\begin{frame}
\frametitle{Ein handfestes Dokument aufbauen}
\framesubtitle{\ldots und dabei die volle Charme von \LaTeX ~erleben!}
\begin{block}{Neue Befehle in diesem Abschnitt}
\begin{itemize}
  \item \color{nounibaredI}\textbackslash input\color{black}\{\}
  \item \color{nounibaredI}\textbackslash tableofcontents\color{black}
  \item \color{nounibaredI}\textbackslash listoffigures\color{black}
  \item \color{nounibaredI}\textbackslash listoftables\color{black}
  \item \color{nounibaredI}\textbackslash vspace\color{black}\{\}
  \item \color{nounibaredI}\textbackslash today\color{black}
\item \color{unibablueI}\textbackslash begin\color{black}\{titlepage\} \ldots \color{unibablueI}\textbackslash end\color{black}\{titlepage\} 
\end{itemize}
\end{block}
\end{frame}

\begin{frame}
\frametitle{Aufbau von einem gr\"o\ss erem Dokument}

\begin{columns}
\begin{column}{.5\textwidth}
\footnotesize
\begin{figure}[t]
\begin{tikzpicture}[dirtree]
\node {main.tex} 
        child {node {command.tex}}
        child {node {titlepage.tex}}
        child {node {introduction.tex}}
        child {node {aufgabe1.tex}}
        child {node {aufgabe2.tex}}
        child {node {aufgabe3.tex}}
        child {node {conclusion.tex}};
\end{tikzpicture}
\end{figure}
\end{column}
\begin{column}{.5\textwidth}
In der main.tex werden alle anderen Dateien zu einem Dokument zusammengefasst. 
Dazu muss man die einzelnen Dateien daf\"ur anpassen.
\end{column}
\end{columns}
\end{frame}

\begin{frame}
\frametitle{\ldots ~und das wollen wir nun machen:}
Es muss \textbf{alles} (einschliesslich) vor und nach \color{unibablueI}\textbackslash begin\color{black}- 
und\color{unibablueI}~\textbackslash end\color{black}\{document\} gel\"oscht werden:\\[5mm]
\begin{columns}
\begin{column}{.47\textwidth}
\sout{\color{nounibaredI}\color{nounibaredI}\textbackslash documentclass\color{black}\color{nounibagreenI}[pdftex]\color{black}\{article\} \\
\color{nounibaredI}\textbackslash usepackage\color{black}\{babel\}\\
\color{nounibaredI}\color{unibablueI}\textbackslash\color{unibablueI}begin\color{black}\color{black}\{document\} }\\
Dieses Dokument kann nun mit dem Befehl \color{nounibaredI}\color{nounibaredI}\textbackslash input\color{nounibaredI}\textbackslash \color{black}\{Dateiname\} in LaTeX eingebunden werden.\color{nounibaredI}\color{nounibaredI}\textbackslash \color{nounibaredI}\textbackslash \color{black}  \\
\sout{\color{nounibaredI}\color{unibablueI}\textbackslash\color{unibablueI}end\color{black}\color{black}\{document\} }
\end{column}
\begin{column}{.47\textwidth}
Mit dem Befehl \color{nounibaredI}\textbackslash input\color{black}\{pfad/zur/datei\} kann man danach
 diese in eine andere {\ttfamily .tex}-Datei einbinden. 

\end{column}
\end{columns}
\end{frame}

\begin{frame}
\frametitle{Verteiltes Arbeiten}
\framesubtitle{\ldots den \"Uberblick behalten}
\begin{columns}{2}
\begin{column}{.6\textwidth}
\image{\textwidth}{image/outline.png}{Texmaker listet die {\ttfamily\color{nounibaredI}\textbackslash input}\color{black}s}{img:outline}
\end{column}
\begin{column}{.3\textwidth}
\begin{alertblock}{Achtung:}
Die Pfadangabe ist immer relativ zur Hauptdatei!
\end{alertblock}
\end{column}
\end{columns}

\end{frame}


\begin{frame}[t]
\frametitle{Exkurs: Titelseite}
\begin{columns}
\begin{column}{0.5\textwidth}
%\begin{titlepage}
\begin{center}
\Huge \LaTeX\\
\vspace{5mm} \LARGE Eine kurze Einführung\\
\vspace{12mm} \Large  Universität Bamberg\\[5mm]
\large 08. April 2015\\
Fachschaft WIAI\normalsize \\
\end{center}
%\end{titlepage}
\end{column}
\begin{column}{0.5\textwidth}
\color{nounibaredI}\color{unibablueI}\textbackslash\color{unibablueI}begin\color{black}\color{black}\{titlepage\} \\\color{black}
\color{nounibaredI}\color{unibablueI}\textbackslash\color{unibablueI}begin\color{black}\color{black}\{center\} \\
\color{nounibaredI}\color{nounibaredI}\textbackslash Huge\color{black} \color{nounibaredI}\color{nounibaredI}\textbackslash LaTeX\color{nounibaredI}\textbackslash \color{nounibaredI}\textbackslash \color{black} \\
\color{nounibaredI}\color{nounibaredI}\textbackslash vspace\color{black}\{5mm\} \color{nounibaredI}\color{nounibaredI}\textbackslash LARGE\color{black} Eine kurze Einführung\color{nounibaredI}\color{nounibaredI}\textbackslash \color{nounibaredI}\textbackslash \color{black} \\
\color{nounibaredI}\color{nounibaredI}\textbackslash vspace\color{black}\{12mm\} \color{nounibaredI}\color{nounibaredI}\textbackslash Large\color{black}  Universität Bamberg\color{nounibaredI}\color{nounibaredI}\textbackslash \color{nounibaredI}\textbackslash \color{black}\color{nounibagreenI}[5mm]\color{black} \\
\color{nounibaredI}\color{nounibaredI}\textbackslash large\color{black} \color{nounibaredI}\color{nounibaredI}\textbackslash today\color{nounibaredI}\textbackslash \color{nounibaredI}\textbackslash \color{black} \\
Fachschaft WIAI\color{nounibaredI}\color{nounibaredI}\textbackslash normalsize\color{black} \color{nounibaredI}\color{nounibaredI}\textbackslash \color{nounibaredI}\textbackslash \color{black} \\
\color{nounibaredI}\color{unibablueI}\textbackslash\color{unibablueI}end\color{black}\color{black}\{center\} \\
\color{nounibaredI}\color{unibablueI}\textbackslash\color{unibablueI}end\color{black}\color{black}\{titlepage\} \\\color{black}
\end{column}
\end{columns}
\end{frame}


\begin{frame}
\frametitle{Verteiltes Arbeiten}
\framesubtitle{Befehle zur Seitennumerierung}
\begin{block}{Seitennumerierung}
\begin{itemize}
\item \color{nounibaredI}\textbackslash thispagestyle\color{black}\{empty\} -- Keine Seitennumerierung
\item \color{nounibaredI}\textbackslash setcounter\color{black}\{page\}\{1\} -- Setzt die Seitennummerierung auf einen bestimmten Wert
\item \color{nounibaredI}\textbackslash pagenumbering\color{black}\{Roman$\mid$roman$\mid$arabic$\mid$Alph$\mid$alph\} -- Definiert die Seitenz\"ahlung
\item \color{nounibaredI}\textbackslash  newpage \color{black}-- Erzeugt eine neue Seite
\end{itemize}
\end{block}
\end{frame}
