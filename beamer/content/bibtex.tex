\section{BibTeX}
\begin{frame}
\frametitle{BibTeX}
\framesubtitle{Add On f"ur \LaTeX}
\begin{exampleblock}{Neue Pakete in diesem Abschnitt}
\begin{itemize}
\item natbib
\end{itemize}
\end{exampleblock}

\begin{block}{Neue Befehle in diesem Abschnitt}
\begin{itemize}
\item \color{nounibaredI}\textbackslash renewcommand\color{black}\{\color{nounibaredI}\textbackslash refname\color{black}\}\{...\}
\item \color{nounibaredI}\textbackslash bibliographystyle\color{black}\{...\}
\item \color{nounibaredI}\textbackslash bibliography\color{black}\{...\}
\item \color{nounibaredI}\textbackslash addcontentsline\color{black}\{toc\}\{section\}\{\color{nounibaredI}\textbackslash refname\color{black}\}
\end{itemize}
\end{block}
\end{frame}

\begin{frame}
\frametitle{BibTeX und \LaTeX ~in Kombination}
Vorteile:
\begin{itemize}
\item Literaturverzeichnis kann in einer vom Dokument unabh"angigen Datei gespeichert werden
\item Speicherung der Daten im BibTeX-Format. Hierbei kann man nach Quellenart unterscheiden, z.B. mit $@$Book, $@$Article usw.
\item Aufnahme der Eintr"age ins Literaturverzeichnis (LVZ) des Dokuments nur dann, wenn die Quelle zitiert wurde
\item Anpassung des Aussehens des LVZ durch unterschiedlichste Style-Sheets leicht m"oglich 
\item Entwurf eigener Style-Sheets zur Gestaltung des LVZ m"oglich
\end{itemize}
Nachteile:
\begin{itemize}
\item Das Erstellen des LVZ mit den gew"unschten Anforderungen ist zum Teil nur erschwert m"oglich
\item Zum Teil "au\ss erst zeitaufw"andig
\end{itemize}
\end{frame}

\begin{frame}
\frametitle{BibTeX und \LaTeX ~in Kombination}
\framesubtitle{vgl. \url{http://de.wikipedia.org/wiki/BibTeX}}
Beispieleintrag:\\[1ex]
$@$BOOK\{\\
Culik93,\\
title = \{Die Welt der Pinguine\},\\
publisher = \{\{BLV\}M"unchen\},\\
year = \{1993\},\\
author = \{B.M. Culik and R. P. Wilson\}\}\\
\vspace*{5mm}
Erkl"arungen zum Eintrag:
\begin{itemize}
\item $@$BOOK - Angabe der Quellenart, hier also ein Buch
\item Culik93 - Definition eines eindeutigen Referenzierungsschl"ussels 
\item title - Titel des Buches
\item publisher - Verlag
\item year - Erscheinungsjahr
\item author - Autor des Buches
\end{itemize}
\end{frame}

\begin{frame}
\frametitle{BibTeX und \LaTeX ~in Kombination}
\begin{tabular}{|p{0.45\textwidth}|p{0.5\textwidth}|}
\hline
\color{nounibaredI}\textbackslash usepackage\color{black} ~natbib & Package natbib zur Darstellung eines alphabetischen LVZ notwendig\\
\hline
\color{nounibaredI}\textbackslash renewcommand\color{black}\{\color{nounibaredI}\textbackslash refname\color{black}\}\newline \{Literaturverzeichnis\} & Anpassung des Namens von Standard Literatur auf Literaturverzeichnis\\
\hline
\color{nounibaredI}\textbackslash bibliographystyle\color{black}\{alphadin\} & Auswahl des Style-Sheets zur Darstellung (hier das Style-Sheet \glqq alphadin\grqq)\\
\hline
\color{nounibaredI}\textbackslash bibliography\color{black}\{bibliography\} & Angabe der Einzubindenden BibTeX-Datei\\
\hline
\color{nounibaredI}\textbackslash addcontentsline\color{black}\{toc\}\{section\}\newline \{\color{nounibaredI}\textbackslash refname\color{black}\} & Aufnahme des LVZ ins Inhatsverzeichnis\\
\hline
\end{tabular}
\end{frame}

\begin{frame}
\frametitle{BibTeX und \LaTeX ~in Kombination}
\framesubtitle{Beispiele für Styles}
\begin{columns}
\begin{column}{.3\textwidth}
\begin{itemize}
\item \textbf{Alphadin-Style}
\vspace*{20mm}
\item \textbf{Abbrvdin-Style}
\end{itemize}
\end{column}
\begin{column}{.7\textwidth}
\image{\textwidth}{image/Styles1.png}{Alphadin und Abbrvdin Style}{img:Styles1}
\end{column}
\end{columns}
\end{frame}

\begin{frame}
\frametitle{BibTeX und \LaTeX ~in Kombination}
\framesubtitle{Beispiele für Styles cont'd}
\begin{columns}
\begin{column}{.3\textwidth}
\begin{itemize}
\item \textbf{Natdin-Style}
\vspace*{20mm}
\item \textbf{Plaindin-Style}
\end{itemize}
\end{column}
\begin{column}{.7\textwidth}
\image{\textwidth}{image/Styles2.png}{Natdin und Plaindin Style}{img:Styles2}
\end{column}
\end{columns}
\end{frame}

\begin{frame}
\frametitle{BibTeX und \LaTeX ~in Kombination}
\framesubtitle{Styles im fertigen Dokument}
\begin{columns}
\begin{column}{.3\textwidth}
\begin{itemize}
\item \textbf{Alphadin-Style}
\vspace*{10mm}
\item \textbf{Abbrvdin-Style}
\vspace*{10mm}
\item \textbf{Natdin-Style}
\vspace*{10mm}
\item \textbf{Plaindin-Style}
\end{itemize}
\end{column}
\begin{column}{.7\textwidth}
\vspace*{5mm}
\image{\textwidth}{image/Styles3.png}{Styles im fertigen Dokument}{img:Styles3}
\end{column}
\end{columns}
\end{frame}