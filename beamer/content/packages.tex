
\section{Formatierung}

\begin{frame}
\frametitle{Ein erstes Anwendungsbeispiel}
\framesubtitle{\"Uberschriften, Inhaltsverzeichnis, einfache Formatierung,
Sonderzeichen}
\begin{block}{Neue Befehle in diesem Abschnitt}
\begin{multicols}{2}
\begin{itemize}
  \item \begin{ttfamily}\color{nounibaredII}\textbackslash usepackage\color{black}\{Paket\}
  \item \color{nounibaredII}\textbackslash befehl\color{nounibagreenI}[evtl\_optionen]\color{black}\{\\Formatierter\_Text\}
  \item \color{unibablueI}\textbackslash begin\color{black}\{Umgebung\}
  \item \color{unibablueI}\textbackslash end\color{black}\{Umgebung\}
  \item \color{nounibaredII}$\backslash\backslash$\color{black}
  \item \color{nounibaredI}\textbackslash newpage\color{black}
  \item \color{unibablueI}\textbackslash sub$^*$section\color{black}\{Titel\}
  \item $\color{nounibaredII}\backslash$\color{nounibaredII}textbf\color{black}\{Text\}
  \item $\color{nounibaredII}\backslash$\color{nounibaredII}textit\color{black}\{Text\}
  \item $\color{nounibaredII}\backslash$\color{nounibaredII}underline\color{black}\{Text\}
  \item \color{nounibaredI}$\color{nounibaredI}\backslash$tiny
  \item \color{nounibaredI}$\color{nounibaredI}\backslash$scriptsize
  \item \color{nounibaredI}$\color{nounibaredI}\backslash$footnotesize
  \item \color{nounibaredI}$\color{nounibaredI}\backslash$normalsize
  \item \color{nounibaredI}$\color{nounibaredI}\backslash$large
  \item \color{nounibaredI}$\color{nounibaredI}\backslash$Large
  \item \color{nounibaredI}$\color{nounibaredI}\backslash$LARGE
  \item \color{nounibaredI}$\color{nounibaredI}\backslash$huge\end{ttfamily}
\end{itemize}
\end{multicols}
\end{block}
\end{frame}


%\begin{frame}
%\frametitle{Ein erstes Anwendungsbeispiel}
%\framesubtitle{Pakete einbinden und Befehle anwenden}
%\begin{itemize}
%  \item Pakete sind Sammlungen von Befehlen oder enthalten z.B. Zeichensätze.\\ Sie werden zu
%  Beginn einer \TeX-Datei angegeben:\\
%  \smallskip
%\textbf{\begin{ttfamily}\color{nounibaredII}\textbackslash usepackage\color{black}\{babel\}
%
%\smallskip
%\end{ttfamily}}
% Einbinden des Paketes „\begin{ttfamily}babel\end{ttfamily}“. (F\"ur Internationalisierung)
%\item Schreibweise von Latex-Befehlen:
%
%\textbf{\begin{ttfamily}\color{nounibaredII}\textbackslash befehl\color{nounibagreenI}[evtl\_optionen]\color{black}\{Formatierter\_Text\}\end{ttfamily}}
%\begin{itemize}
%  \item in \begin{ttfamily}\{\}\end{ttfamily} stehen immer notwendige Parameter bzw. Text
% \item in \begin{ttfamily}[ ]\end{ttfamily} stehen (falls vorhanden)
% zus"atzliche, optionale Parameter
% \item zum Beispiel:
%
%
%\begin{ttfamily}
%\color{nounibaredII}\textbackslash documentclass\color{nounibagreenI}[a4paper,12pt,pdftex,ngerman]\color{black}\{article\}
%\end{ttfamily}
%\end{itemize}
%\end{frame}

\begin{frame}
\frametitle{Ein erstes Anwendungsbeispiel}
\framesubtitle{Befehle cont't}
\begin{columns}
\begin{column}{0.6\textwidth}
\begin{ttfamily}\scriptsize
\color{nounibaredI}\color{nounibaredI}\textbackslash documentclass\color{black}\color{nounibagreenI}[a4paper, pdftex, ngerman]\color{black}\{article\} \\
\color{nounibaredI}\color{nounibaredI}\textbackslash usepackage\color{black}\color{nounibagreenI}[utf8]\color{black}\{inputenc\} \\
\color{nounibaredI}\color{nounibaredI}\textbackslash usepackage\color{black}\color{nounibagreenI}[T1]\color{black}\{fontenc\} \\
\color{nounibaredI}\color{unibablueI}\textbackslash\color{unibablueI}begin\color{black}\color{black}\{document\} \\
Das ist ein einfaches Minidokument \\
ohne Besonderheiten. Zeilenumbrüche \\
funktionieren immer automatisch! \\
Mehrere \\
Leerzeichen hintereinander werden  \\
zu einem zusammengefasst. \\
Getrennt wird auch automatisch.\color{nounibaredI}\color{nounibaredI}\textbackslash \color{nounibaredI}\textbackslash \color{black} \\
Mit zwei Backslashs beginnt eine neue \\
Zeile.\color{nounibaredI}\color{nounibaredI}\textbackslash \color{nounibaredI}\textbackslash \color{black} \\
Ein neuer Absatz entsteht durch eine \\
leere Zeile. \\
\color{nounibaredI}\color{unibablueI}\textbackslash\color{unibablueI}end\color{black}\color{black}\{document\} \\

\end{ttfamily}
\end{column}

\begin{column}{0.4\textwidth}
Es gibt verschiedene Arten von Dokumenten.\\ Hier wird die Dokumentenart
\begin{ttfamily}article\end{ttfamily} verwendet (weiter möglich:
\begin{ttfamily}book\end{ttfamily} und \begin{ttfamily}report\end{ttfamily}) In
\begin{ttfamily}[]\end{ttfamily} steht die Papiergröße und die Schriftgröße des
Standardtextes.\\
%! TODO!
\end{column}
\end{columns}
\end{frame}

\begin{frame}
\frametitle{Ein erstes Anwendungsbeispiel}
\framesubtitle{Befehle cont't}
\begin{columns}
\begin{column}{0.6\textwidth}
\begin{ttfamily}\scriptsize
\color{nounibaredI}\color{nounibaredI}\textbackslash documentclass\color{black}\color{nounibagreenI}[a4paper, pdftex, ngerman]\color{black}\{article\} \\
\color{nounibaredI}\color{nounibaredI}\textbackslash usepackage\color{black}\color{nounibagreenI}[utf8]\color{black}\{inputenc\} \\
\color{nounibaredI}\color{nounibaredI}\textbackslash usepackage\color{black}\color{nounibagreenI}[T1]\color{black}\{fontenc\} \\
\color{nounibaredI}\color{unibablueI}\textbackslash\color{unibablueI}begin\color{black}\color{black}\{document\} \\
Das ist ein einfaches Minidokument \\
ohne Besonderheiten. Zeilenumbrüche \\
funktionieren immer automatisch! \\
Mehrere \\
Leerzeichen hintereinander werden  \\
zu einem zusammengefasst. \\
Getrennt wird auch automatisch.\color{nounibaredI}\color{nounibaredI}\textbackslash \color{nounibaredI}\textbackslash \color{black} \\
Mit zwei Backslashs beginnt eine neue \\
Zeile.\color{nounibaredI}\color{nounibaredI}\textbackslash \color{nounibaredI}\textbackslash \color{black} \\
Ein neuer Absatz entsteht durch eine \\
leere Zeile. \\
\color{nounibaredI}\color{unibablueI}\textbackslash\color{unibablueI}end\color{black}\color{black}\{document\} \\

 \normalsize
\end{ttfamily}
\end{column}
\begin{column}{0.4\textwidth}
\begin{ttfamily}\textbf{\color{unibablueI}\textbackslash begin\color{black}\{Umgebung\}}\end{ttfamily}\\
Es beginnt eine neue Umgebung, hier das eigentliche Dokument.\\[5mm]

\begin{ttfamily}\textbf{\color{unibablueI}\textbackslash end\color{black}\{Umgebung\}}\end{ttfamily}\\
Die mit \begin{ttfamily}\textbf{\color{unibablueI}\textbackslash begin}\color{black}\{\}\end{ttfamily}
eingeleitete Umgebung ist hier zu Ende.\\[5mm]

\begin{ttfamily}\textbf{\color{nounibaredII}$\backslash\backslash$}\color{black}
~Zeilenumbruch\end{ttfamily}\\
\end{column}
\end{columns}
\end{frame}



\begin{frame}
\frametitle{Ein erstes Anwendungsbeispiel}
\framesubtitle{Pakete}
\begin{columns}
\begin{column}{0.6\textwidth}
\begin{ttfamily}\scriptsize
\color{nounibaredI}\color{nounibaredI}\textbackslash documentclass\color{black}\color{nounibagreenI}[a4paper, pdftex, ngerman]\color{black}\{article\} \\
\color{nounibaredI}\color{nounibaredI}\textbackslash usepackage\color{black}\color{nounibagreenI}[utf8]\color{black}\{inputenc\} \\
\color{nounibaredI}\color{nounibaredI}\textbackslash usepackage\color{black}\color{nounibagreenI}[T1]\color{black}\{fontenc\} \\
\color{nounibaredI}\color{unibablueI}\textbackslash\color{unibablueI}begin\color{black}\color{black}\{document\} \\
Das ist ein einfaches Minidokument \\
ohne Besonderheiten. Zeilenumbrüche \\
funktionieren immer automatisch! \\
Mehrere \\
Leerzeichen hintereinander werden  \\
zu einem zusammengefasst. \\
Getrennt wird auch automatisch.\color{nounibaredI}\color{nounibaredI}\textbackslash \color{nounibaredI}\textbackslash \color{black} \\
Mit zwei Backslashs beginnt eine neue \\
Zeile.\color{nounibaredI}\color{nounibaredI}\textbackslash \color{nounibaredI}\textbackslash \color{black} \\
Ein neuer Absatz entsteht durch eine \\
leere Zeile. \\
\color{nounibaredI}\color{unibablueI}\textbackslash\color{unibablueI}end\color{black}\color{black}\{document\} \\

 \normalsize
\end{ttfamily}
\end{column}
\begin{column}{0.4\textwidth}
\begin{ttfamily}\textbf{ngerman}\end{ttfamily}\\
Für Deutschland typische Formatierungen und (Trenn)-Regeln werden verwendet.\\[5mm]

\begin{ttfamily}\textbf{inputenc}\end{ttfamily}\\
Definiert den Zeichen-\\satz, der verwendet werden soll. Es sollte immer
\begin{ttfamily}UTF-8\end{ttfamily} verwendet werden, weil er universal auf
allen Betriebssystemen l"auft.\\
\end{column}
\end{columns}
\end{frame}

\begin{frame}
\frametitle{Exkurs}
\framesubtitle{Zeichenkodierungen}
\begin{columns}
\begin{column}{0.6\textwidth}
\image{\textwidth}{image/utf8.png}{UTF-8 im Texmaker}{img:utf8}

\end{column}
\begin{column}{0.4\textwidth}
Wird ein Dokument geöffnet, wird automatisch der richtige Zeichensatz benutzt.
Beim Erstellen neuer Dokumente wird die Datei in dem Format gespeichert, die im
 Editor voreingestellt ist. In den Texmaker-Einstelllungen muss derselbe Zeichensatz verwendet werden, der
auch im erstellten LaTeX-Dokument verwendet wird.\\
\end{column}
\end{columns}
\textbf{Bei Gruppenarbeiten muss jedes Mitglied zwingend \underline{UTF-8} im
Editor einstellen, sonst ist Ärger so gut wie vorprogrammiert!} (Kaputte
Umlaute, Kompilierungsfehler uvm., wenn es nicht nur „Windows“-User gibt.)
\end{frame}

\begin{frame}
\frametitle{Ein erstes Anwendungsbeispiel}
\framesubtitle{Als .PDF}
\begin{columns}
\begin{column}{0.5\textwidth}
\begin{ttfamily}\scriptsize
\color{nounibaredI}\color{nounibaredI}\textbackslash documentclass\color{black}\color{nounibagreenI}[a4paper, pdftex, ngerman]\color{black}\{article\} \\
\color{nounibaredI}\color{nounibaredI}\textbackslash usepackage\color{black}\color{nounibagreenI}[utf8]\color{black}\{inputenc\} \\
\color{nounibaredI}\color{nounibaredI}\textbackslash usepackage\color{black}\color{nounibagreenI}[T1]\color{black}\{fontenc\} \\
\color{nounibaredI}\color{unibablueI}\textbackslash\color{unibablueI}begin\color{black}\color{black}\{document\} \\
Das ist ein einfaches Minidokument \\
ohne Besonderheiten. Zeilenumbrüche \\
funktionieren immer automatisch! \\
Mehrere \\
Leerzeichen hintereinander werden  \\
zu einem zusammengefasst. \\
Getrennt wird auch automatisch.\color{nounibaredI}\color{nounibaredI}\textbackslash \color{nounibaredI}\textbackslash \color{black} \\
Mit zwei Backslashs beginnt eine neue \\
Zeile.\color{nounibaredI}\color{nounibaredI}\textbackslash \color{nounibaredI}\textbackslash \color{black} \\
Ein neuer Absatz entsteht durch eine \\
leere Zeile. \\
\color{nounibaredI}\color{unibablueI}\textbackslash\color{unibablueI}end\color{black}\color{black}\{document\} \\

 \normalsize
\end{ttfamily}
\end{column}

\begin{column}{0.5\textwidth}
\image{\textwidth}{image/minidocument.png}{Der Code von der linken Seite als .pdf.}{listing:minidocument}
\end{column}
\end{columns}
\end{frame}



\begin{frame}
\frametitle{Abschnitte}
\framesubtitle{Kapitelmarken}
\begin{columns}
\begin{column}{0.5\textwidth}
\begin{ttfamily}\scriptsize
\color{nounibaredI}\color{nounibaredI}\textbackslash documentclass\color{black}\color{nounibagreenI}[a4paper, pdftex, 12pt, ngerman]\color{black}\{article\} \\
\color{nounibaredI}\color{nounibaredI}\textbackslash usepackage\color{black}\color{nounibagreenI}[utf8]\color{black}\{inputenc\} \\
\color{nounibaredI}\color{nounibaredI}\textbackslash usepackage\color{black}\color{nounibagreenI}[T1]\color{black}\{fontenc\} \\
\color{nounibaredI}\color{nounibaredI}\textbackslash usepackage\color{black}\{babel\} \\
\color{nounibaredI}\color{unibablueI}\textbackslash\color{unibablueI}begin\color{black}\color{black}\{document\} \\
\color{nounibaredI}\color{nounibaredI}\textbackslash tableofcontents\color{black} \\
\color{nounibaredI}\color{unibablueI}\textbackslash\color{unibablueI}section\color{black}\color{black}\{Kapitel 1\} \\
Hier kommt der erste Teil. \\
\color{nounibaredI}\color{unibablueI}\textbackslash\color{unibablueI}subsection\color{black}\color{black}\{Unterkapitel 1\} \\
Das erste Unterkapitel. \\
\color{nounibaredI}\color{unibablueI}\textbackslash\color{unibablueI}subsection\color{black}\color{black}\{Unterkapitel 2\} \\
Und noch ein Unterkapitel. \\
\color{nounibaredI}\color{unibablueI}\textbackslash\color{unibablueI}subsubsection\color{black}\color{black}\{Unterunterkapitel 1\} \\
Das ist ein Unterkapitel von einem Unterkapitel. \\
\color{nounibaredI}\color{unibablueI}\textbackslash\color{unibablueI}end\color{black}\color{black}\{document\} \\

\end{ttfamily}
\end{column}
\begin{column}{0.5\textwidth}
\begin{ttfamily}\color{nounibaredI}\textbackslash newpage\color{black}\end{ttfamily}\\
Seitenumbruch\\[3mm]
\begin{ttfamily}\color{nounibaredI}\textbackslash tableofcontents\color{black}\end{ttfamily}\\
Automatisches Inhaltsverzeichnis\\[3mm]
\begin{ttfamily}\color{unibablueI}\textbackslash section\color{black}\{Titel\}\end{ttfamily}\\
Ein neuer Abschnitt mit dem in \begin{ttfamily}\{\}\end{ttfamily} angegebenen Titel
beginnt.\\[3mm]
\begin{ttfamily}\color{unibablueI}\textbackslash subsection\color{black}\{Titel\}\end{ttfamily}\\
Ein Unterabschnitt.\\[3mm]
\begin{ttfamily}\color{unibablueI}\textbackslash subsubsection\color{black}\{Titel\}\end{ttfamily}\\
Noch eine Ebene darunter.\\
\end{column}
\end{columns}
\end{frame}

\begin{frame}
\frametitle{Abschnitte}
\framesubtitle{Kapitelmarken .PDF}
\begin{columns}
\begin{column}{0.45\textwidth}
\begin{ttfamily}\scriptsize
\color{nounibaredI}\color{nounibaredI}\textbackslash documentclass\color{black}\color{nounibagreenI}[a4paper, pdftex, 12pt, ngerman]\color{black}\{article\} \\
\color{nounibaredI}\color{nounibaredI}\textbackslash usepackage\color{black}\color{nounibagreenI}[utf8]\color{black}\{inputenc\} \\
\color{nounibaredI}\color{nounibaredI}\textbackslash usepackage\color{black}\color{nounibagreenI}[T1]\color{black}\{fontenc\} \\
\color{nounibaredI}\color{nounibaredI}\textbackslash usepackage\color{black}\{babel\} \\
\color{nounibaredI}\color{unibablueI}\textbackslash\color{unibablueI}begin\color{black}\color{black}\{document\} \\
\color{nounibaredI}\color{nounibaredI}\textbackslash tableofcontents\color{black} \\
\color{nounibaredI}\color{unibablueI}\textbackslash\color{unibablueI}section\color{black}\color{black}\{Kapitel 1\} \\
Hier kommt der erste Teil. \\
\color{nounibaredI}\color{unibablueI}\textbackslash\color{unibablueI}subsection\color{black}\color{black}\{Unterkapitel 1\} \\
Das erste Unterkapitel. \\
\color{nounibaredI}\color{unibablueI}\textbackslash\color{unibablueI}subsection\color{black}\color{black}\{Unterkapitel 2\} \\
Und noch ein Unterkapitel. \\
\color{nounibaredI}\color{unibablueI}\textbackslash\color{unibablueI}subsubsection\color{black}\color{black}\{Unterunterkapitel 1\} \\
Das ist ein Unterkapitel von einem Unterkapitel. \\
\color{nounibaredI}\color{unibablueI}\textbackslash\color{unibablueI}end\color{black}\color{black}\{document\} \\

\end{ttfamily}
\end{column}
\begin{column}{0.55\textwidth}
\image{\textwidth}{image/chapters.png}{Die Kapitel werden automatisch mitgez\"ahlt}{img:chapters}
\end{column}
\end{columns}
\end{frame}


\begin{frame}
\frametitle{Abschnitte}
\framesubtitle{Part \& Chapter}
Neben \begin{ttfamily}\color{unibablueI}\textbackslash section\color{black}\{\},
\color{unibablueI}\textbackslash subsection\color{black}\{\}\end{ttfamily}, und \begin{ttfamily}\color{unibablueI}\textbackslash subsubsection\color{black}\{\}\end{ttfamily} gibt es auch noch den Befehl
 \begin{ttfamily}\color{unibablueI}\textbackslash part\color{black}\{\}\end{ttfamily} welcher einen größeren Teil definiert.
\begin{ttfamily}\color{unibablueI}\textbackslash part\color{black}\{\}\end{ttfamily} füllt eine ganze eigene Seite.\\
Neben dem Dokumentypen {\ttfamily article} existieren f"ur Flie\ss
textdokumente noch weitere wie {\ttfamily book} und {\ttfamily report}.\\
Bei {\ttfamily book} wird in der Regel zwischen linker und rechter Seite
unterschieden, wobei die sich z.B. darin unterscheiden, ob die Seitenzahl links oder rechts steht, bzw. was sonst noch in der Kopf- oder Fußzeile stehen kann.
In {\ttfamily book} und {\ttfamily report} gibt es noch den
Gliederungsbefehl \begin{ttfamily}\color{unibablueI}\textbackslash chapter\color{black}\{\}\end{ttfamily}.

%\begin{columns}
%\begin{column}{0.5\textwidth}
%CODE
%\end{column}
%\begin{column}{0.5\textwidth}
%OUTPUT
%\end{column}
%\end{columns}
\end{frame}


\begin{frame}
\frametitle{Formatierungen}
\framesubtitle{FettKursivUnterstrichen}
\begin{columns}
\begin{column}{0.45\textwidth}
\begin{ttfamily}\scriptsize
\color{nounibaredI}\color{nounibaredI}\textbackslash documentclass\color{black}\color{nounibagreenI}[a4paper, pdftex, 12pt, ngerman]\color{black}\{article\} \\
\color{nounibaredI}\color{nounibaredI}\textbackslash usepackage\color{black}\color{nounibagreenI}[utf8]\color{black}\{inputenc\} \\
\color{nounibaredI}\color{nounibaredI}\textbackslash usepackage\color{black}\color{nounibagreenI}[T1]\color{black}\{fontenc\} \\
\color{nounibaredI}\color{unibablueI}\textbackslash\color{unibablueI}begin\color{black}\color{black}\{document\} \\
Unter anderem folgende M"oglichkeiten:\color{nounibaredI}\color{nounibaredI}\textbackslash \color{nounibaredI}\textbackslash \color{black} \\
\color{nounibaredI}\color{nounibaredI}\textbackslash textbf\color{black}\{fetter\}\color{nounibaredI}\color{nounibaredI}\textbackslash \color{nounibaredI}\textbackslash \color{black} \\
\color{nounibaredI}\color{nounibaredI}\textbackslash textit\color{black}\{kursiver\}\color{nounibaredI}\color{nounibaredI}\textbackslash \color{nounibaredI}\textbackslash \color{black} \\
\color{nounibaredI}\color{nounibaredI}\textbackslash underline\color{black}\{unterstrichener\}\color{nounibaredI}\color{nounibaredI}\textbackslash \color{nounibaredI}\textbackslash \color{black} \\
\color{nounibaredI}\color{nounibaredI}\textbackslash underline\color{black}\{\color{nounibaredI}\color{nounibaredI}\textbackslash textbf\color{black}\{unterstrichen und fett\}\}\color{nounibaredI}\color{nounibaredI}\textbackslash \color{nounibaredI}\textbackslash \color{black} \\
\color{nounibaredI}\color{unibablueI}\textbackslash\color{unibablueI}end\color{black}\color{black}\{document\} \\

\end{ttfamily}
\end{column}
\begin{column}{0.55\textwidth}
Unter anderem folgende M"oglichkeiten:\\[3mm]
\textbf{fetter}\\
\textit{kursiver}\\
\underline{unterstrichener}\\
\underline{\textbf{unterstrichen und fett}}
%\input{formats_pdf.tex}
\begin{block}{Textformatierungen}
\begin{ttfamily}$\color{nounibaredII}\backslash$\color{nounibaredII}textbf\color{black}\{Text\}\end{ttfamily}
fetter Text\\
\begin{ttfamily}$\color{nounibaredII}\backslash$\color{nounibaredII}textit\color{black}\{Text\}\end{ttfamily}
kursiver Text\\
\begin{ttfamily}$\color{nounibaredII}\backslash$\color{nounibaredII}underline\color{black}\{Text\}\end{ttfamily}
unterstrichen
\end{block}
\end{column}
\end{columns}
\end{frame}

\begin{frame}
\frametitle{Formatierungen}
\framesubtitle{Schriftgr"o\ss e}
\begin{columns}
\begin{column}{0.5\textwidth}
\begin{ttfamily}\scriptsize
\color{nounibaredI}\color{nounibaredI}\textbackslash tiny\color{black}~unlesbarer Text \\
\color{nounibaredI}\color{nounibaredI}\textbackslash scriptsize\color{black}~sehr kleiner Text \\
\color{nounibaredI}\color{nounibaredI}\textbackslash footnotesize\color{black}~Fussnotengröße \\
\color{nounibaredI}\color{nounibaredI}\textbackslash small\color{black}~klein \\
\color{nounibaredI}\color{nounibaredI}\textbackslash normalsize\color{black}~Standardgröße \\
\color{nounibaredI}\color{nounibaredI}\textbackslash large\color{black}~ größer \\
\color{nounibaredI}\color{nounibaredI}\textbackslash Large\color{black}~noch größer \\
\color{nounibaredI}\color{nounibaredI}\textbackslash LARGE\color{black}~sehr groß \\
\color{nounibaredI}\color{nounibaredI}\textbackslash huge\color{black}~riesig \\
\color{nounibaredI}\color{nounibaredI}\textbackslash Huge\color{black}~gigantisch \\

\end{ttfamily}
\end{column}
\begin{column}{0.5\textwidth}
\rm \tiny ~unlesbarer Text \\
\scriptsize ~sehr kleiner Text \\
\footnotesize ~Fu\ss notengr\"o\ss e\\
\normalsize ~Standardgr\"o\ss e \\
\large ~gr\"o\ss er \\
\Large ~noch gr\"o\ss er \\
\LARGE ~sehr Gro\ss \\
\huge  ~riesig \\
\end{column}
\end{columns}
\end{frame}

\begin{frame}
\frametitle{Formatierungen}
\framesubtitle{FettKursivunterstrichen}
\begin{columns}
\begin{column}{0.5\textwidth}
\begin{ttfamily}\scriptsize\color{nounibaredII}\textbackslash documentclass\color{nounibagreenI}[a4paper, pdftex, 12pt, ngerman]\color{black}\{article\}\\[3mm] 
$\color{nounibaredII}\backslash$\color{nounibaredII}usepackage\color{nounibagreenI}[utf8]\color{black}\{inputenc\}\\
$\color{nounibaredII}\backslash$\color{nounibaredII}usepackage\color{nounibagreenI}[T1]\color{black}\{fontenc\}\\
$\color{nounibaredII}\backslash$\color{nounibaredII}usepackage\color{nounibagreenI}[iso]\color{black}\{umlaute\}\\
$\color{nounibaredII}\backslash$\color{nounibaredII}usepackage\color{black}\{babel\}\\
\color{gray}\% NEU NEU NEU\\
$\color{nounibaredII}\backslash$\color{nounibaredII}usepackage\color{black}\{eurosym\}\\
$\color{unibablueI}\backslash$\color{unibablueI}begin\color{black}\{document\}\\
$\color{nounibaredII}\backslash$\color{nounibaredII}textit\color{black}\{Einige
Sonderzeichen:\}\\
\color{nounibaredII}\textbackslash \% \textbackslash \$ \textbackslash \& \textbackslash \{ \textbackslash \}
\textbackslash \_ \textbackslash \# \textbackslash S \textbackslash copyright\\
\textbackslash slash \~ ~ \color{unibayellowI}\$\color{nounibaredII}$\color{nounibaredII}\backslash$backslash\color{unibayellowI}\$\color{nounibaredII}  ~\textbackslash euro \\

$\color{nounibaredII}\backslash$\color{nounibaredII}textit\color{black}\{Binde-\slash
Gedanken-\slash Trennstriche:\} \\
- -- --- \color{unibayellowI}\$\color{black}-\color{unibayellowI}\$\color{black} (letzteres mathematisches Minus) \\

$\color{nounibaredII}\backslash$\color{nounibaredII}textit\color{black}\{Anf"uhrungszeichen aus \begin{ttfamily}ngerman\end{ttfamily}:\} \\
\color{nounibaredII}\textbackslash glqq \textbackslash grqq \textbackslash flqq \textbackslash frqq\\
\color{unibablueI}\textbackslash end\color{black}\{document\}
\end{ttfamily}
\end{column}
\begin{column}{0.5\textwidth}
\textit{Einige Sonderzeichen:}    \\
\% \$ \& \{ \} \_ \# \S ~ \copyright \slash ~ \textbackslash  \euro \\

\textit{Binde-\slash Gedanken-\slash Trennstriche:} \\
- -- --- $-$ (letzteres mathematisches Minus) \\

\textit{Anführungszeichen aus (n)german:} \\
\glqq \grqq \flqq \frqq\\[5mm]
Für das \euro -Zeichen wird das Package \begin{ttfamily}eurosym\end{ttfamily}
benötigt.\\

\end{column}
\end{columns}
\medskip
\footnotesize Sonderzeichen müssen mit dem '\color{nounibaredI}\textbackslash \color{black}' eingeführt werden.
Manchmal, z.B. in \"Uberschriften m\"ussen Umlaute des Pakets ngerman mit \grqq a \grqq o
\grqq u und das \ss ~mit \color{nounibaredI} \textbackslash ss \color{black}gebildet werden, ansonsten reicht es das Packet {\ttfamily babel} einzubinden.
\end{frame}
