\section{Tabellen}

\begin{frame}
\frametitle{Tabellen}
\framesubtitle{Einf\"ugen von Tabellen}

\begin{exampleblock}{Neue Pakete in diesem Abschnitt}
\begin{itemize}
\item longtable
\end{itemize}
\end{exampleblock}

\begin{block}{Neue Befehle in diesem Abschnitt}
\begin{itemize}
\item \color{unibablueI}\textbackslash begin\color{black}\{tabular\} \dots
~\color{unibablueI}\textbackslash end\color{black}\{tabular\}
\item \color{unibablueI}\textbackslash begin\color{black}\{table\} \dots
~\color{unibablueI}\textbackslash end\color{black}\{table\}
\item \color{unibablueI}\textbackslash begin\color{black}\{longtable\} \dots
~\color{unibablueI}\textbackslash end\color{black}\{longtable\}
\item \color{unibablueI}\textbackslash begin\color{black}\{tabbing\} \dots
~\color{unibablueI}\textbackslash end\color{black}\{tabbing\}
\item \color{nounibaredI}$|$\color{black}
\item \color{nounibaredI}\& \color{black}
\item \color{nounibaredI}\textbackslash hline\color{black}
\item \color{nounibaredI}\textbackslash multicolumn\color{black}\{\}\{\}\{\}
\end{itemize}
\end{block}

\end{frame}

%-----------------------------------------------------------------------------

\begin{frame}
\frametitle{Tabellen}
\framesubtitle{Einfachste Tabelle "`tabular"':}
%\textbf{Aufbau:}\\[2mm]
%\color{unibablueI}\begin{ttfamily}\textbackslash begin\color{black}\{table\}\color{nounibagreenI}[Position]\color{black}\\
%\color{unibablueI}\textbackslash begin\color{black}\{tabular\}\{\textit{Spaltendefinitionen}\}\\
%\textit{Tabelleninhalt}\\
%\color{unibablueI}\textbackslash end\color{black}\{tabular\}\\
%\color{nounibaredI}\textbackslash caption\color{black}\{Untertitel\}\\
%\color{nounibaredI}\textbackslash label\color{black}\{tab:bsptab1\}\\
%\color{unibablueI}\textbackslash end\color{black}\{table\}\\
%~\\
%\end{ttfamily}
\textbf{Aufbau:}\\
\begin{ttfamily}
\color{unibablueI}\textbackslash begin\color{black}\{tabular\}\{\textit{Spaltendefinitionen}\}\\
~~\textit{Tabelleninhalt}\\
\color{unibablueI}\textbackslash end\color{black}\{tabular\}\\
\end{ttfamily}

%\begin{block}{Reminder: Positionsangaben f\"ur die meisten \LaTeX -- Umgebungen}
%\color{nounibagreenI}[h]\color{black}~oder \color{nounibagreenI}[H]\color{black}~= hier an dieser Stelle\\
%\color{nounibagreenI}[t]\color{black}~= oben auf der Seite\\ 
%\color{nounibagreenI}[b]\color{black}~= unten auf der Seite\\ 
%\color{nounibagreenI}[p]\color{black}~= Platzierung auf der einer eigenen Seite
%\end{block}
\medskip
\textbf{Spaltendefinition:}\\
Hier wird bestimmt wie die einzelnen Spalten ausgerichtet sein sollen
und wie die senkrechten Tabellenlinien gesetzt werden sollen:\\
\begin{tabbing}[H]p{Spaltenbreite}
xxx\=\kill
\textbf{Befehle:}\\
l \>= linksbündige Spalte\\
c \>= zentrierte Spalte\\
r \>= rechtsbündige Spalte\\
p\{Spaltenbreite\} \>= eine linksbündige Spalte mit bestimmter Spaltenbreite\\
\color{nounibaredI}$|$\color{black} \>= setzt eine senkrechte Tabellenlinie an
dieser Stelle\\
\end{tabbing}
\end{frame}


%-----------------------------------------------------------------------------

\begin{frame}
\frametitle{Tabellen}
\framesubtitle{Einfachste Tabelle "`tabular"':}
\medskip
\textbf{Aufbau:}\\
\begin{ttfamily}
\color{unibablueI}\textbackslash begin\color{black}\{tabular\}\{\textit{Spaltendefinitionen}\}\\
~~\textit{Tabelleninhalt}\\
\color{unibablueI}\textbackslash end\color{black}\{tabular\}\\
\end{ttfamily}
\medskip
\textbf{Tabelleninhalt:}\\
Hier werden die definierten Spalten mit Inhalt gefüllt und horizontal getrennt.
\begin{tabbing}[H]p{Spaltenbreite}
xxx\=\kill
\textbf{Befehle:}\\
\color{nounibaredI}\&\color{black} \>= dient zur horizontalen Trennung von Zellen\\
\color{nounibaredI}\textbackslash \textbackslash \color{black} \>=  neue Zeile\\
\color{nounibaredI}\textbackslash hline\color{black} \>= setzt eine waagerechte Tabellenlinie\\[2mm]
\color{nounibaredI}\textbackslash multicolumn\color{black}\{Spaltenzahl\}\{Spaltenausrichtung\}\{Text\}\\[2mm]
\>= Verbindet beliebig viele Spalten miteinander.\\
\end{tabbing}
\end{frame}

%----------------------------------------------------------------------------


%\begin{frame}
%\frametitle{Tabellen}
%\framesubtitle{Spaltendefinitionen}
%Hier wird bestimmt wie die einzelnen Spalten ausgerichtet sein sollen
%und wie die senkrechten Tabellenlinien gesetzt werden sollen:\\[3mm]
%\begin{tabbing}[H]p{Spaltenbreite}xxx\=\kill
%\textbf{Befehle:}\\
%l \>= linksbündige Spalte\\
%c \>= zentrierte Spalte\\
%r \>= rechtsbündige Spalte\\
%p\{Spaltenbreite\} \>= eine linksbündige Spalte mit bestimmter Spaltenbreite\\
%\color{nounibaredI}$|$\color{black} \>= setzt eine senkrechte Tabellenlinie an
%dieser Stelle\\
%\end{tabbing}
%\end{frame}

%-----------------------------------------------------------------------------

%\begin{frame}
%\frametitle{Tabellen}
%\framesubtitle{Ein Blick ins Innere}
%\begin{ttfamily}
%\color{nounibaredI}\color{unibablueI}\textbackslash\color{unibablueI}begin\color{black}\color{black}\{tabular\}\{c\color{nounibaredI}|\color{black}p\{40mm\}\color{nounibaredI}|\color{black}lr\color{nounibaredI}|\color{black}c\} \\
\color{nounibaredI}\color{nounibaredI}\textbackslash multicolumn\color{black}\{5\}\{c\}\{E-Sports Championship Franconia\} \color{nounibaredI}\color{nounibaredI}\textbackslash \color{nounibaredI}\textbackslash \color{black} \\
\color{nounibaredI}\color{nounibaredI}\textbackslash hline\color{black} \\
\color{nounibaredI}\color{nounibaredI}\textbackslash hline\color{black} \\
Number \color{nounibaredI}\&  \color{black}Place \color{nounibaredI}\&  \color{black}Player 1 \color{nounibaredI}\&  \color{black}Player 2 \color{nounibaredI}\&  \color{black}Result \color{nounibaredI}\color{nounibaredI}\textbackslash \color{nounibaredI}\textbackslash \color{black} \\
\color{nounibaredI}\color{nounibaredI}\textbackslash hline\color{black} \\
1 \color{nounibaredI}\&  \color{black}Nürnberg \color{nounibaredI}\&  \color{black}Wolf \color{nounibaredI}\&  \color{black}Lamm \color{nounibaredI}\&  \color{black}23:10 \color{nounibaredI}\color{nounibaredI}\textbackslash \color{nounibaredI}\textbackslash \color{black} \\
\color{nounibaredI}\color{nounibaredI}\textbackslash hline\color{black} \\
2 \color{nounibaredI}\&  \color{black}Bamberg \color{nounibaredI}\&  \color{black}Meyer \color{nounibaredI}\&  \color{black}Beyer \color{nounibaredI}\color{nounibaredI}\textbackslash \color{nounibaredI}\textbackslash \color{black} \\
\color{nounibaredI}\color{nounibaredI}\textbackslash hline\color{black} \\
3 \color{nounibaredI}\&  \color{black}Zirndorf \color{nounibaredI}\&  \color{black}Brandst. \color{nounibaredI}\&  \color{black}Brauer \color{nounibaredI}\&  \color{black}21:21\color{nounibaredI}\color{nounibaredI}\textbackslash \color{nounibaredI}\textbackslash \color{black} \\
\color{nounibaredI}\color{nounibaredI}\textbackslash hline\color{black} \\
\color{nounibaredI}\color{unibablueI}\textbackslash\color{unibablueI}end\color{black}\color{black}\{tabular\} \\

%\end{ttfamily}
%\end{frame}
%
%\begin{frame}
%\frametitle{Tabellen}
%\framesubtitle{Tabelleninhalt}
%Hier werden die definierten Spalten mit Inhalt gefüllt.\\[3mm]
%\begin{tabbing}[H]p{Spaltenbreite}xxx\=\kill
%\textbf{Befehle:}\\
%\color{nounibaredI}\&\color{black} \>= dient zur horizontalen Trennung von Zellen\\
%\color{nounibaredI}\textbackslash \textbackslash \color{black} \>=  neue Zeile\\
%\color{nounibaredI}\textbackslash hline\color{black} \>= setzt eine waagerechte Tabellenlinie\\[2mm]
%\color{nounibaredI}\textbackslash multicolumn\color{black}\{Spaltenzahl\}\{Spaltenausrichtung\}\{Text\}\\[2mm]
%\>= Verbindet beliebig viele Spalten miteinander.\\
%\end{tabbing}
%\end{frame}

%-----------------------------------------------------------------------------

\begin{frame}[t]

\frametitle{Tabellen}
\framesubtitle{Beispiel Tabular}
\begin{footnotesize}
\begin{ttfamily}
\color{nounibaredI}\color{unibablueI}\textbackslash\color{unibablueI}begin\color{black}\color{black}\{tabular\}\{c\color{nounibaredI}|\color{black}p\{40mm\}\color{nounibaredI}|\color{black}lr\color{nounibaredI}|\color{black}c\} \\
\color{nounibaredI}\color{nounibaredI}\textbackslash multicolumn\color{black}\{5\}\{c\}\{E-Sports Championship Franconia\} \color{nounibaredI}\color{nounibaredI}\textbackslash \color{nounibaredI}\textbackslash \color{black} \\
\color{nounibaredI}\color{nounibaredI}\textbackslash hline\color{black} \\
\color{nounibaredI}\color{nounibaredI}\textbackslash hline\color{black} \\
Number \color{nounibaredI}\&  \color{black}Place \color{nounibaredI}\&  \color{black}Player 1 \color{nounibaredI}\&  \color{black}Player 2 \color{nounibaredI}\&  \color{black}Result \color{nounibaredI}\color{nounibaredI}\textbackslash \color{nounibaredI}\textbackslash \color{black} \\
\color{nounibaredI}\color{nounibaredI}\textbackslash hline\color{black} \\
1 \color{nounibaredI}\&  \color{black}Nürnberg \color{nounibaredI}\&  \color{black}Wolf \color{nounibaredI}\&  \color{black}Lamm \color{nounibaredI}\&  \color{black}23:10 \color{nounibaredI}\color{nounibaredI}\textbackslash \color{nounibaredI}\textbackslash \color{black} \\
\color{nounibaredI}\color{nounibaredI}\textbackslash hline\color{black} \\
2 \color{nounibaredI}\&  \color{black}Bamberg \color{nounibaredI}\&  \color{black}Meyer \color{nounibaredI}\&  \color{black}Beyer \color{nounibaredI}\color{nounibaredI}\textbackslash \color{nounibaredI}\textbackslash \color{black} \\
\color{nounibaredI}\color{nounibaredI}\textbackslash hline\color{black} \\
3 \color{nounibaredI}\&  \color{black}Zirndorf \color{nounibaredI}\&  \color{black}Brandst. \color{nounibaredI}\&  \color{black}Brauer \color{nounibaredI}\&  \color{black}21:21\color{nounibaredI}\color{nounibaredI}\textbackslash \color{nounibaredI}\textbackslash \color{black} \\
\color{nounibaredI}\color{nounibaredI}\textbackslash hline\color{black} \\
\color{nounibaredI}\color{unibablueI}\textbackslash\color{unibablueI}end\color{black}\color{black}\{tabular\} \\

\end{ttfamily}
\end{footnotesize}

\begin{tabular}{c|p{40mm}|lr|c}
\multicolumn{5}{c}{E-Sports Meisterschaft Franken}
 \\
\hline
\hline
Nummer & Ort & Spieler 1 & Spieler 2 & Ergebnis \\
\hline
1 & N\"urnberg & Wolf & Lamm & 23:10 \\
\hline
2 & Bamberg & Meyer & Beyer & \\
\hline
3 & Zirndorf & Brandst. & Brauer & 21:21 \\
\hline
\end{tabular}
\end{frame}

%-----------------------------------------------------------------------------


\begin{frame}[t]

\frametitle{Tabellen}
\framesubtitle{Weiteres Beispiel}

\begin{block}{Hinweis}
Zeilenumbrüche in Zellen von \color{nounibaredI}\textbackslash p\color{black}\{\} Spalten können mit \color{nounibaredI}\textbackslash newline\color{black}~ erzielt werden. Es bricht dort aber auch automatisch um.
\end{block}

\medskip

\begin{footnotesize}
\begin{ttfamily}
\color{nounibaredI}\color{unibablueI}\textbackslash\color{unibablueI}begin\color{black}\color{black}\{tabular\}\{\color{nounibaredI}|\color{black}c\color{nounibaredI}|\color{black}p\{20mm\}\color{nounibaredI}|\color{black}c\color{nounibaredI}|\color{black}c\color{nounibaredI}|\color{black}\} \\
\color{nounibaredI}\color{nounibaredI}\textbackslash hline\color{black} \\
1 \color{nounibaredI}\&  \color{black}Long \color{nounibaredI}\color{nounibaredI}\textbackslash newline\color{black}~ Text \color{nounibaredI}\&  \color{black}Michael \color{nounibaredI}\&  \color{black}Träger \color{nounibaredI}\color{nounibaredI}\textbackslash \color{nounibaredI}\textbackslash \color{black} \\
\color{nounibaredI}\color{nounibaredI}\textbackslash hline\color{black} \\
2 \color{nounibaredI}\&  \color{black}Very Long Text \color{nounibaredI}\&  \color{black}Valentin \color{nounibaredI}\&  \color{black}Barth \color{nounibaredI}\color{nounibaredI}\textbackslash \color{nounibaredI}\textbackslash \color{black} \\
\color{nounibaredI}\color{nounibaredI}\textbackslash hline\color{black} \\
\color{nounibaredI}\color{unibablueI}\textbackslash\color{unibablueI}end\color{black}\color{black}\{tabular\} \\
\end{ttfamily}
\end{footnotesize}

\medskip

\begin{tabular}{|c|p{20mm}|c|c|}
\hline
1 & Long \newline Text & Michael & Träger \\
\hline
2 & Very Long Text & Valentin & Barth \\
\hline
\end{tabular}

\end{frame}

%-----------------------------------------------------------------------------



\begin{frame}
\frametitle{Tabellen}
\framesubtitle{Beschriften und Positionieren}
%\textbf{Aufbau:}\\[2mm]
Zum Beschriften müssen "`tabulars"' in einer "`table"'-Umgebung liegen (Wie bei Grafiken)\\ 

\medskip

\color{unibablueI}\begin{ttfamily}\textbackslash begin\color{black}\{table\}\color{nounibagreenI}[Position]\color{black}\\
~~\color{unibablueI}\textbackslash begin\color{black}\{tabular\}\{\textit{Spaltendefinitionen}\}\\
~~~~\textit{Tabelleninhalt}\\
~~\color{unibablueI}\textbackslash end\color{black}\{tabular\}\\
~~\color{nounibaredI}\textbackslash caption\color{black}\{Untertitel\}\\
~~\color{nounibaredI}\textbackslash label\color{black}\{tab:bsptab1\}\\
\color{unibablueI}\textbackslash end\color{black}\{table\}\\
\end{ttfamily}

\medskip

\begin{block}{Reminder: Positionsangaben f\"ur die meisten \LaTeX -- Umgebungen}
\color{nounibagreenI}[h]\color{black}~oder \color{nounibagreenI}[H]\color{black}~= hier an dieser Stelle\\
\color{nounibagreenI}[t]\color{black}~= oben auf der Seite\\ 
\color{nounibagreenI}[b]\color{black}~= unten auf der Seite\\ 
\color{nounibagreenI}[p]\color{black}~= Platzierung auf der einer eigenen Seite
\end{block}
\end{frame}

%-----------------------------------------------------------------------------


\begin{frame}{Tabellen}
\framesubtitle{Longtable -- Tabelle mit Seitenumbruch}
\bigskip
Bei „tabular“ wird die Tabelle auf einer Seite angezeigt. Wenn sie nicht draufpasst, wird sie abgeschnitten.\\
F"ur Tabellen, die länger als eine Seite sind, wird eine Tabelle benötigt die eine Trennung der Tabelle vornimmt.\\
\textbf{Lösung: {\ttfamily longtable}}\\
{\ttfamily longtable} ermöglicht den Seitenumbruch in der Tabelle. Ausserdem ist {\ttfamily longtable} eine eigene Umgebung, braucht deshalb keine {\ttfamily table}-Umgebung mehr!\\[3mm]

\begin{ttfamily}
\color{unibablueI}\textbackslash begin\color{black}\{longtable\}\{\textit{Spaltendefinitionen}\}\\
\textit{Tabelleninhalt}\\
\color{nounibaredI}\textbackslash caption\color{black}\{Untertitel\}\\
\color{nounibaredI}\textbackslash label\color{black}\{tab:bsptab2\}\\
\color{unibablueI}\textbackslash end\color{black}\{longtable\}
\end{ttfamily}
\end{frame}

%-----------------------------------------------------------------------------

\begin{frame}
\frametitle{Einrückungen durch „tabbing“}
\begin{block}{Steuerung}
\begin{itemize}
\item[\begin{ttfamily}\color{nounibaredI}\textbackslash =\end{ttfamily}]\color{black}eine Tabulatorstelle setzen 
\item[\begin{ttfamily}\color{nounibaredI}\textbackslash $>$\end{ttfamily}]eine Tabulatorstelle ansteuern
\end{itemize}
\end{block}

\begin{columns}
\begin{column}{.5\textwidth}
\begin{ttfamily}{\scriptsize
\color{nounibaredI}\color{nounibaredI}\textbackslash documentclass\color{black}\{article\} \\
\color{nounibaredI}\color{unibablueI}\textbackslash\color{unibablueI}begin\color{black}\color{black}\{document\} \\
\color{nounibaredI}\color{unibablueI}\textbackslash\color{unibablueI}begin\color{black}\color{black}\{tabbing\} \\
Employ\color{nounibaredI}\color{nounibaredI}\textbackslash \color{black}=ee:\color{nounibaredI}\color{nounibaredI}\textbackslash \color{nounibaredI}\textbackslash \color{black} \\
A  \color{nounibaredI}\color{nounibaredI}\textbackslash \color{black}> Daniel\color{nounibaredI}\color{nounibaredI}\textbackslash \color{nounibaredI}\textbackslash \color{black} \\
B  \color{nounibaredI}\color{nounibaredI}\textbackslash \color{black}> Martin\color{nounibaredI}\color{nounibaredI}\textbackslash \color{nounibaredI}\textbackslash \color{black} \\
C  \color{nounibaredI}\color{nounibaredI}\textbackslash \color{black}> Linus\color{nounibaredI}\color{nounibaredI}\textbackslash \color{nounibaredI}\textbackslash \color{black} \\
xxx\color{nounibaredI}\color{nounibaredI}\textbackslash \color{black}=xxx\color{nounibaredI}\color{nounibaredI}\textbackslash \color{black}=xxxxxxx\color{nounibaredI}\color{nounibaredI}\textbackslash kill\color{black} \\
\color{nounibaredI}\color{nounibaredI}\textbackslash \color{black}> Committees\color{nounibaredI}\color{nounibaredI}\textbackslash \color{nounibaredI}\textbackslash \color{black} \\
\color{nounibaredI}\color{nounibaredI}\textbackslash \color{black}>\color{nounibaredI}\color{nounibaredI}\textbackslash \color{black}> Tests\color{nounibaredI}\color{nounibaredI}\textbackslash \color{nounibaredI}\textbackslash \color{black} \\
\color{nounibaredI}\color{nounibaredI}\textbackslash \color{black}>\color{nounibaredI}\color{nounibaredI}\textbackslash \color{black}> Mails\color{nounibaredI}\color{nounibaredI}\textbackslash \color{nounibaredI}\textbackslash \color{black} \\
\color{nounibaredI}\color{unibablueI}\textbackslash\color{unibablueI}end\color{black}\color{black}\{tabbing\} \\
\color{nounibaredI}\color{unibablueI}\textbackslash\color{unibablueI}end\color{black}\color{black}\{document\} \\
}
\end{ttfamily}
\end{column}
\begin{column}{.5\textwidth}
\begin{tabbing}
Mitarb\=eiter:\\
A  \> Daniel\\
B  \> Martin\\
C  \> Linus\\
xxx\=xxx\=xxxxxxx\kill
\> Gremien\\
\>\> Klausuren\\
\>\> Emails\\
\end{tabbing}
\end{column}
\end{columns}

Durch den Befehl\begin{ttfamily} \color{nounibaredI}\textbackslash kill\color{black}\end{ttfamily} wird der Zeileninhalt nicht angezeigt. Dadurch können Formatierungen vorgenommen werden, ohne den zugehörigen Text anzuzeigen.
\end{frame}
