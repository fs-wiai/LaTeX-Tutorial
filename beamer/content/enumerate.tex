\section{Aufz\"ahlungen}
\begin{frame}
\frametitle{Aufzählungen}

\begin{block}{Neue Befehle in diesem Abschnitt}
\begin{itemize}
\item \color{unibablueI}\textbackslash begin\color{black}\{itemize\} \ldots \color{unibablueI}\textbackslash end\color{black}\{itemize\} 
\item \color{unibablueI}\textbackslash begin\color{black}\{enumerate\} \ldots \color{unibablueI}\textbackslash end\color{black}\{enumerate\} 
\item \color{nounibaredI}\textbackslash item\color{black}
\end{itemize}
\end{block}
\end{frame}

\begin{frame}
\frametitle{Aufzählungen}
\framesubtitle{Spiegelstrichlisten}

\begin{columns}
\begin{column}{.5\textwidth}
\begin{ttfamily}
\color{nounibaredI}\color{nounibaredI}\textbackslash documentclass\color{black}\{article\} \\
\color{nounibaredI}\color{unibablueI}\textbackslash\color{unibablueI}begin\color{black}\color{black}\{document\} \\
\color{nounibaredI}\color{unibablueI}\textbackslash\color{unibablueI}begin\color{black}\color{black}\{itemize\} \\
\color{nounibaredI}\color{nounibaredI}\textbackslash item\color{black} erster Stichpunkt \\
\color{nounibaredI}\color{nounibaredI}\textbackslash item\color{black} zweiter Stichpunkt \\
\color{nounibaredI}\color{nounibaredI}\textbackslash item\color{black} dritter Stichpunkt \\
\color{nounibaredI}\color{nounibaredI}\textbackslash item\color{black} letzter Stichpunkt \\
\color{nounibaredI}\color{unibablueI}\textbackslash\color{unibablueI}end\color{black}\color{black}\{itemize\} \\
\color{nounibaredI}\color{unibablueI}\textbackslash\color{unibablueI}end\color{black}\color{black}\{document\} \\
\end{ttfamily}
\end{column}
\begin{column}{.5\textwidth}
\begin{itemize}
\item erster Stichpunkt
\item zweiter Stichpunkt
\item dritter Stichpunkt
\item letzter Stichpunkt
\end{itemize}
\end{column}
\end{columns}
\bigskip

Die einzelnen Stichpunkte werden innerhalb der „itemize“-Umgebung durch den Befehl \begin{ttfamily}\color{nounibaredI}\textbackslash item\color{black}\end{ttfamily} gekennzeichnet.
\end{frame}

\begin{frame}
\frametitle{Aufzählungen}
\framesubtitle{Verschachtelung}

\begin{columns}
\begin{column}{.5\textwidth}
\begin{ttfamily}
\color{nounibaredI}\color{nounibaredI}\textbackslash documentclass\color{black}\{article\} \\
\color{nounibaredI}\color{unibablueI}\textbackslash\color{unibablueI}begin\color{black}\color{black}\{document\} \\
\color{nounibaredI}\color{unibablueI}\textbackslash\color{unibablueI}begin\color{black}\color{black}\{itemize\} \\
\color{nounibaredI}\color{nounibaredI}\textbackslash item \color{black} erster Stichpunkt \\
\color{nounibaredI}\color{nounibaredI}\textbackslash item \color{black} zweiter Stichpunkt \\
\color{nounibaredI}\color{unibablueI}\textbackslash\color{unibablueI}begin\color{black}\color{black}\{itemize\} \\
\color{nounibaredI}\color{nounibaredI}\textbackslash item \color{black} erster Unterpunkt \\
\color{nounibaredI}\color{nounibaredI}\textbackslash item \color{black} zweiter Unterpunkt \\
\color{nounibaredI}\color{unibablueI}\textbackslash\color{unibablueI}end\color{black}\color{black}\{itemize\} \\
\color{nounibaredI}\color{nounibaredI}\textbackslash item \color{black} dritter Stichpunkt \\
\color{nounibaredI}\color{nounibaredI}\textbackslash item \color{black} letzter Stichpunkt \\
\color{nounibaredI}\color{unibablueI}\textbackslash\color{unibablueI}end\color{black}\color{black}\{itemize\} \\
\color{nounibaredI}\color{unibablueI}\textbackslash\color{unibablueI}end\color{black}\color{black}\{document\} \\
\end{ttfamily}
\end{column}
\begin{column}{.5\textwidth}
\begin{itemize}
\item erster Stichpunkt
\item zweiter Stichpunkt
\begin{itemize}
\item erster Unterpunkt
\item zweiter Unterpunkt
\end{itemize}
\item dritter Stichpunkt
\item letzter Stichpunkt
\end{itemize}
\end{column}
\end{columns}
\bigskip
Auf diese Weise kann man Unterpunkte bis auf 4 Ebenen tief schachteln.
\end{frame}


\begin{frame}
\frametitle{Aufzählungen}
\framesubtitle{Nummerierungen}

\begin{columns}
\begin{column}{.5\textwidth}
\begin{ttfamily}
\color{nounibaredI}\color{nounibaredI}\textbackslash documentclass\color{black}\{article\} \\
\color{nounibaredI}\color{unibablueI}\textbackslash\color{unibablueI}begin\color{black}\color{black}\{document\} \\
\color{nounibaredI}\color{unibablueI}\textbackslash\color{unibablueI}begin\color{black}\color{black}\{enumerate\} \\
\color{nounibaredI}\color{nounibaredI}\textbackslash item \color{black} first bullet item \\
\color{nounibaredI}\color{unibablueI}\textbackslash\color{unibablueI}begin\color{black}\color{black}\{enumerate\} \\
\color{nounibaredI}\color{nounibaredI}\textbackslash item \color{black} first subitem \\
\color{nounibaredI}\color{nounibaredI}\textbackslash item \color{black} second subitem \\
\color{nounibaredI}\color{unibablueI}\textbackslash\color{unibablueI}end\color{black}\color{black}\{enumerate\} \\
\color{nounibaredI}\color{nounibaredI}\textbackslash item \color{black} second bullet item \\
\color{nounibaredI}\color{nounibaredI}\textbackslash item \color{black} and so forth \\
\color{nounibaredI}\color{unibablueI}\textbackslash\color{unibablueI}end\color{black}\color{black}\{enumerate\} \\
\color{nounibaredI}\color{unibablueI}\textbackslash\color{unibablueI}end\color{black}\color{black}\{document\} \\
\end{ttfamily}
\end{column}
\begin{column}{.5\textwidth}
\begin{enumerate}
\item erstens
\begin{enumerate}
\item erster Unterpunkt
\item zweiter Unterpunkt
\end{enumerate}
\item zweitens
\item usw.
\end{enumerate}
\end{column}
\end{columns}
\bigskip
Auch hier werden die einzelnen Punkte durch den Befehl \begin{ttfamily}\color{nounibaredI}\textbackslash item\color{black}\end{ttfamily} gekennzeichnet. 
Schachtelungen können wieder bis zu 4 Ebenen tief sein.
\end{frame}

\begin{frame}
\frametitle{Gemischte Aufzählungen?}
\framesubtitle{Geht Alles!}

\begin{columns}
\begin{column}{.5\textwidth}
\begin{ttfamily}
\color{nounibaredI}\color{unibablueI}\textbackslash\color{unibablueI}begin\color{black}\color{black}\{enumerate\} \\
\color{nounibaredI}\color{nounibaredI}\textbackslash item\color{black} erstens \\
\color{nounibaredI}\color{nounibaredI}\textbackslash item\color{black} \color{nounibaredI}\color{unibablueI}\textbackslash\color{unibablueI}begin\color{black}\color{black}\{itemize\} \\
\color{nounibaredI}\color{nounibaredI}\textbackslash item\color{black} erster Unterpunkt \\
\color{nounibaredI}\color{nounibaredI}\textbackslash item\color{black} zweiter Unterpunkt \\
\color{nounibaredI}\color{unibablueI}\textbackslash\color{unibablueI}end\color{black}\color{black}\{itemize\} \\
\color{nounibaredI}\color{nounibaredI}\textbackslash item\color{black} drittens \\
\color{nounibaredI}\color{unibablueI}\textbackslash\color{unibablueI}begin\color{black}\color{black}\{enumerate\} \\
\color{nounibaredI}\color{nounibaredI}\textbackslash item\color{black} Auch ich zähle! \\
\color{nounibaredI}\color{unibablueI}\textbackslash\color{unibablueI}end\color{black}\color{black}\{enumerate\} \\
\color{nounibaredI}\color{nounibaredI}\textbackslash item\color{black} usw. \\
\color{nounibaredI}\color{unibablueI}\textbackslash\color{unibablueI}end\color{black}\color{black}\{enumerate\} \\
\end{ttfamily}
\end{column}
\begin{column}{.5\textwidth}
\begin{enumerate}
\item erstens
\item \begin{itemize}
\item erster Unterpunkt
\item zweiter Unterpunkt
\end{itemize}
\item drittens
\begin{enumerate}
\item Auch ich z"ahle!
\end{enumerate}
\item usw.
\end{enumerate}
\end{column}
\end{columns}
\bigskip
Die Darstellung der jeweiligen Symbole kann mit \color{nounibaredI}\textbackslash item\color{nounibagreenI}[]\color{black}~angepasst werden. 
\end{frame}