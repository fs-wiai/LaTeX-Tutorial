\section{Code}

\begin{frame}
\frametitle{Code}
\framesubtitle{Programmcode darstellen}

\begin{exampleblock}{Neue Pakete in diesem Abschnitt}
\begin{multicols}{3}
\begin{itemize}
\item verbatim
\item listings
\item color
\end{itemize}
\end{multicols}
\end{exampleblock}

\begin{block}{Neue Befehle in diesem Abschnitt}
%\begin{multicols}{2}
\begin{itemize}
\item \color{unibablueI}\textbackslash begin\color{black}\{verbatim\} \dots
~\color{unibablueI}\textbackslash end\color{black}\{verbatim\}
\item \color{unibablueI}\textbackslash begin\color{black}\{lstlisting\} \dots
~\color{unibablueI}\textbackslash end\color{black}\{lstlisting\}
\item \color{nounibaredI}\textbackslash color\color{black}\{\}
\item \color{nounibaredI}\textbackslash lstset\color{black}\{\}
%TODO mehr?
\end{itemize}
%\end{multicols}
\end{block}

\end{frame}

%-----------------------------------------------------------------------------

\begin{frame}[fragile]
\frametitle{Code}
\framesubtitle{Unformatierte Texte \& Codeabschnitte}
Die Verbatim Umgebung:\\
\begin{columns}
\begin{column}{.5\textwidth}
\begin{ttfamily}
\begin{tabbing}
x\=\kill\\
\>\color{unibablueI}\textbackslash begin\color{black}\{verbatim\}\\
\>Dieser Satz erscheint\\
\>so im Text.\\
\>Auch Befehle wie\\
\>\textbackslash textbf\{werden\}\\
\>nicht interpretiert.\\
\>\color{unibablueI}\textbackslash end\color{black}\{verbatim\}\\
\end{tabbing}
\end{ttfamily}

\end{column}
\begin{column}{.5\textwidth}
\begin{verbatim}
Dieser Satz erscheint
so im Text.
Auch Befehle wie
\textbf{werden}
nicht interpretiert.
\end{verbatim}
\end{column}
\end{columns}
\end{frame}

%-------------------------------------------------------------------------------

\begin{frame}[fragile]
\frametitle{Code}
\framesubtitle{Unformatierte Texte \& Codeabschnitte}
\vspace{3mm}
%Die Verbatim Umgebung (Paket: verbatim):\\
\scriptsize
\lstset{language=Java, commentstyle=\color{green}}
\begin{lstlisting}
public static void printNumber(int n)
{
    for (int i = 0; i < n; i++)
    {
        // print the current number
        System.out.println("Number: " + i);
    }
}
\end{lstlisting}

\footnotesize
\vspace{-2mm}

\begin{ttfamily}
\begin{tabbing}
xx\=xx\=\kill\\
\color{nounibaredI}\textbackslash usepackage\color{black}\{listings\}\\
\color{nounibaredI}\textbackslash usepackage\color{black}\{color\}\\
\color{nounibaredI}\textbackslash lstset\color{black}\{language=Java, commentstyle=\color{nounibaredI}\textbackslash color\color{black}\{green\}\}\\
\color{unibablueI}\textbackslash begin\color{black}\{lstlisting\}\\
public static void printNumber(int n)\\
\{\\
\>for (int i = 0; i < n; i++)\\
\>\{\\
\>\>// print the current number\\
\>\>System.out.println(\verb|"|Number: \verb|"| + i);\\
\>\}\\
\}\\
\color{unibablueI}\textbackslash end\color{black}\{lstlisting\}\\
\end{tabbing}
\end{ttfamily}
\normalsize
\end{frame}



%TODO hinweis auf weitere Formatierungsmöglichkeiten -> syntax hilighting (siehe Internet)

