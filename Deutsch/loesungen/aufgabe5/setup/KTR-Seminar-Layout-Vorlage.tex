\documentclass[a4paper, pdftex, ngerman, 11pt]{article}
%===============================================================================
% Zweck: KTR-Seminar-Vorlage in Anlehung an G. Wirtz, Lehrstuhl Praktische Informatik
%===============================================================================
%===============================================================================
% zentrale Layout-Angaben und Befehle
%===============================================================================
%
\usepackage{babel}
\usepackage[utf8]{inputenc}
\usepackage{fancyhdr}
\usepackage[T1]{fontenc}						
\usepackage{color}
\usepackage{amsmath}
\usepackage{amsfonts}
\usepackage{float}
\usepackage{longtable}
\usepackage{multirow}							%Package zum korrekten Einfuegen von Bildern!
\usepackage[hyphens]{url}
%%   Zur Gestaltung des Textes zu einem Hypertext   %%
\usepackage{hyperref}
\usepackage{listings}
\definecolor{darkblue}{rgb}{0,.05,.54}
\hypersetup{colorlinks=true, breaklinks=true, linkcolor=darkblue, menucolor=darkblue, urlcolor=darkblue, citecolor=darkblue, filecolor=darkblue}
\urlstyle{same}

\usepackage{amsmath,amssymb,ifthen}
\usepackage{graphicx}
%
\if pdf
\usepackage[pdftex,bookmarksopen,bookmarksnumbered,pdfborder=0]{hyperref}
\pdfcompresslevel=9
\else
\usepackage{url}
\fi
%\usepackage[dvips]{rotating}
%
% ausf\"{u}hrlichere Fehlermeldungen
\errorcontextlines=999
%
% Page-Layout: A4 aus Header
% Alternative
%\setlength\headheight{14pt}
%\setlength\topmargin{-15,4mm}
%\setlength\oddsidemargin{-0,4mm}
%\setlength\evensidemargin{-0,4mm}
%\setlength\textwidth{160mm}
%\setlength\textheight{252mm}
%
% Absatzeinstellungen
\setlength\parindent{0mm}
\setlength\parskip{2ex}
%
%
% Erstellung von Abk\"{u}rzungsverzeichnis
\newcommand{\abbrev}[2]{#1 & #2\\}
\newcommand{\abkuerzungen}{
\section*{Abk\"{u}rzungsverzeichnis}
\hspace{2ex}
\begin{tabular}{ll}
\input{abkuerzungen.tex}
\end{tabular}
}
%
% Einbindung eines Bildes mit angegebener Breite
% #1 = label f\"{u}r \ref-Verweise
% #2 = Name des Bildes ohne Endung relativ zu Bilder-Verzeichnis
% #3 = Beschriftung
% #4 = Breite des Bildes im Dokument in cm
\newcommand{\bildw}[4]{%
  \begin{figure}[htb]%
    \centering
    \includegraphics[width=#4cm]{Bilder/#2}%
    \vskip -0.3cm%
    \caption{#3}%
    \vskip -0,2cm%
    \label{#1}%
  \end{figure}%
}
%
% Einbindung eines Bildes mit Seitenbreite
% #1 = label f\"{u}r \ref-Verweise
% #2 = Name des Bildes ohne Endung relativ zu Bilder-Verzeichnis
% #3 = Beschriftung
\newcommand{\bild}[3]{%
  \begin{figure}[htb]%
    \centering%
    \includegraphics[width=\textwidth]{Bilder/#2}%
    \vskip -0.3cm%
    \caption{#3}%
    \vskip -0,2cm%
    \label{#1}%
  \end{figure}%
}
%
\numberwithin{equation}{section}
%
%===============================================================================
% zentrale Layout-Angaben und Befehle
%===============================================================================
%
