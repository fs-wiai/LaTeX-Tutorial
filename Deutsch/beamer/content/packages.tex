
\section{Formatierung}

\begin{frame}
\frametitle{Ein erstes Anwendungsbeispiel}
\framesubtitle{\"Uberschriften, Inhaltsverzeichnis, einfache Formatierung,
Sonderzeichen}
\begin{block}{Neue Befehle in diesem Abschnitt}
\begin{multicols}{2}
\begin{itemize}
  \item \begin{ttfamily}\color{nounibaredI}\textbackslash usepackage\color{black}\{Paket\}
  \item \color{nounibaredI}\textbackslash befehl\color{nounibagreenI}[evtl\_optionen]\color{black}\{\\Formatierter Text\}
  \item \color{unibablueI}\textbackslash begin\color{black}\{Umgebung\}
  \item \color{unibablueI}\textbackslash end\color{black}\{Umgebung\}
  \item \color{nounibaredI}$\backslash\backslash$\color{black}
  \item \color{nounibaredI}\textbackslash newpage\color{black}
  \item \color{unibablueI}\textbackslash sub$^*$section\color{black}\{Titel\}
  \item $\color{nounibaredII}\backslash$\color{nounibaredI}textbf\color{black}\{Text\}
  \item $\color{nounibaredII}\backslash$\color{nounibaredI}textit\color{black}\{Text\}
  \item $\color{nounibaredII}\backslash$\color{nounibaredI}underline\color{black}\{Text\}
  \item \color{nounibaredI}$\color{nounibaredI}\backslash$tiny
  \item \color{nounibaredI}$\color{nounibaredI}\backslash$scriptsize
  \item \color{nounibaredI}$\color{nounibaredI}\backslash$footnotesize
  \item \color{nounibaredI}$\color{nounibaredI}\backslash$normalsize
  \item \color{nounibaredI}$\color{nounibaredI}\backslash$large
  \item \color{nounibaredI}$\color{nounibaredI}\backslash$Large
  \item \color{nounibaredI}$\color{nounibaredI}\backslash$LARGE
  \item \color{nounibaredI}$\color{nounibaredI}\backslash$huge\end{ttfamily}
\end{itemize}
\end{multicols}
\end{block}
\end{frame}

%-------------------------------------------------------

%TODO in erweiterung einbauen?
%\begin{frame}
%\frametitle{Ein erstes Anwendungsbeispiel}
%\framesubtitle{Pakete einbinden und Befehle anwenden}
%\begin{itemize}
%  \item Pakete sind Sammlungen von Befehlen oder enthalten z.B. Zeichensätze.\\ Sie werden zu
%  Beginn einer \TeX-Datei angegeben:\\
%  \smallskip
%\textbf{\begin{ttfamily}\color{nounibaredII}\textbackslash usepackage\color{black}\{babel\}
%
%\smallskip
%\end{ttfamily}}
% Einbinden des Paketes „\begin{ttfamily}babel\end{ttfamily}“. (F\"ur Internationalisierung)
%\item Schreibweise von Latex-Befehlen:
%
%\textbf{\begin{ttfamily}\color{nounibaredII}\textbackslash befehl\color{nounibagreenI}[evtl\_optionen]\color{black}\{Formatierter\_Text\}\end{ttfamily}}
%\begin{itemize}
%  \item in \begin{ttfamily}\{\}\end{ttfamily} stehen immer notwendige Parameter bzw. Text
% \item in \begin{ttfamily}[ ]\end{ttfamily} stehen (falls vorhanden)
% zus"atzliche, optionale Parameter
% \item zum Beispiel:
%
%
%\begin{ttfamily}
%\color{nounibaredII}\textbackslash documentclass\color{nounibagreenI}[a4paper,12pt,pdftex,ngerman]\color{black}\{article\}
%\end{ttfamily}
%\end{itemize}
%\end{frame}

%-------------------------------------------------------

\begin{frame}
\frametitle{Ein erstes Anwendungsbeispiel}
\framesubtitle{Als .PDF}
\begin{columns}
\begin{column}{0.55\textwidth}
\begin{ttfamily}\footnotesize
\color{nounibaredI}\color{nounibaredI}\textbackslash documentclass\color{black}\color{nounibagreenI}[a4paper, pdftex, ngerman]\color{black}\{article\} \\
\color{nounibaredI}\color{nounibaredI}\textbackslash usepackage\color{black}\color{nounibagreenI}[utf8]\color{black}\{inputenc\} \\
\color{nounibaredI}\color{nounibaredI}\textbackslash usepackage\color{black}\color{nounibagreenI}[T1]\color{black}\{fontenc\} \\
\color{nounibaredI}\color{unibablueI}\textbackslash\color{unibablueI}begin\color{black}\color{black}\{document\} \\
Das ist ein einfaches Minidokument \\
ohne Besonderheiten. Zeilenumbrüche \\
funktionieren immer automatisch! \\
Mehrere \\
Leerzeichen hintereinander werden  \\
zu einem zusammengefasst. \\
Getrennt wird auch automatisch.\color{nounibaredI}\color{nounibaredI}\textbackslash \color{nounibaredI}\textbackslash \color{black} \\
Mit zwei Backslashs beginnt eine neue \\
Zeile.\color{nounibaredI}\color{nounibaredI}\textbackslash \color{nounibaredI}\textbackslash \color{black} \\
Ein neuer Absatz entsteht durch eine \\
leere Zeile. \\
\color{nounibaredI}\color{unibablueI}\textbackslash\color{unibablueI}end\color{black}\color{black}\{document\} \\

 \normalsize
\end{ttfamily}
\end{column}

\begin{column}{0.45\textwidth}
\image{\textwidth}{image/minidocument.png}{Der Beispielcode als .PDF.}{listing:minidocument}
\end{column}
\end{columns}
\end{frame}

%-------------------------------------------------------

\begin{frame}
\frametitle{Ein erstes Anwendungsbeispiel}
\framesubtitle{Befehle cont'd}
\begin{columns}
\begin{column}{0.55\textwidth}
\begin{ttfamily}\footnotesize
\color{nounibaredI}\color{nounibaredI}\textbackslash documentclass\color{black}\color{nounibagreenI}[a4paper, pdftex, ngerman]\color{black}\{article\} \\
\color{nounibaredI}\color{nounibaredI}\textbackslash usepackage\color{black}\color{nounibagreenI}[utf8]\color{black}\{inputenc\} \\
\color{nounibaredI}\color{nounibaredI}\textbackslash usepackage\color{black}\color{nounibagreenI}[T1]\color{black}\{fontenc\} \\
\color{nounibaredI}\color{unibablueI}\textbackslash\color{unibablueI}begin\color{black}\color{black}\{document\} \\
Das ist ein einfaches Minidokument \\
ohne Besonderheiten. Zeilenumbrüche \\
funktionieren immer automatisch! \\
Mehrere \\
Leerzeichen hintereinander werden  \\
zu einem zusammengefasst. \\
Getrennt wird auch automatisch.\color{nounibaredI}\color{nounibaredI}\textbackslash \color{nounibaredI}\textbackslash \color{black} \\
Mit zwei Backslashs beginnt eine neue \\
Zeile.\color{nounibaredI}\color{nounibaredI}\textbackslash \color{nounibaredI}\textbackslash \color{black} \\
Ein neuer Absatz entsteht durch eine \\
leere Zeile. \\
\color{nounibaredI}\color{unibablueI}\textbackslash\color{unibablueI}end\color{black}\color{black}\{document\} \\

\end{ttfamily}
\end{column}

\begin{column}{0.45\textwidth}
\color{nounibaredI}\color{nounibaredI}\textbackslash documentclass\color{black}\color{nounibagreenI}[options]\color{black}\{type\} \\
Es gibt verschiedene Arten von Dokumenten.\\ Hier wird die Dokumentenart
\begin{ttfamily}article\end{ttfamily} verwendet.\\
In \begin{ttfamily}[]\end{ttfamily} steht die Papiergröße, die Schriftgröße des
Standardtextes und die Sprache.\\
%! TODO!
\end{column}
\end{columns}
\end{frame}


%-------------------------------------------------------


\begin{frame}
\frametitle{Dokumentarten}
\framesubtitle{Exkurs}

\begin{itemize}

\item Für einfache Dokumente, Artikel:\\
\textbf{scrartcl}, \textbf{article}\\

\item Für komplexere Dokumente, Bachelor-/Masterarbeiten:\\
\textbf{scrreprt}, \textbf{report}\\

\item Für Bücher:\\
\textbf{scrbook}, \textbf{book}\\
{\ttfamily book} unterscheidet zwischen linker und rechter Seite, wobei Unterschiede z. B. die Position der Seitenzahl links oder rechts sein können.

\item Für Präsentationen:\\
\textbf{beamer}, \textbf{seminar}, \textbf{texpower}\\
\vspace{0.5cm}

\end{itemize}

Im europäischen Raum sind die Dokumentklassen des \textbf{KOMA-Script} üblich. Die dazugehörigen Klassen beginnen mit \texttt{scr}.\\
\end{frame}

%-------------------------------------------------------

\begin{frame}
\frametitle{Ein erstes Anwendungsbeispiel}
\framesubtitle{Befehle (Forts.)}
\begin{columns}
\begin{column}{0.55\textwidth}
\begin{ttfamily}\footnotesize
\color{nounibaredI}\color{nounibaredI}\textbackslash documentclass\color{black}\color{nounibagreenI}[a4paper, pdftex, ngerman]\color{black}\{article\} \\
\color{nounibaredI}\color{nounibaredI}\textbackslash usepackage\color{black}\color{nounibagreenI}[utf8]\color{black}\{inputenc\} \\
\color{nounibaredI}\color{nounibaredI}\textbackslash usepackage\color{black}\color{nounibagreenI}[T1]\color{black}\{fontenc\} \\
\color{nounibaredI}\color{unibablueI}\textbackslash\color{unibablueI}begin\color{black}\color{black}\{document\} \\
Das ist ein einfaches Minidokument \\
ohne Besonderheiten. Zeilenumbrüche \\
funktionieren immer automatisch! \\
Mehrere \\
Leerzeichen hintereinander werden  \\
zu einem zusammengefasst. \\
Getrennt wird auch automatisch.\color{nounibaredI}\color{nounibaredI}\textbackslash \color{nounibaredI}\textbackslash \color{black} \\
Mit zwei Backslashs beginnt eine neue \\
Zeile.\color{nounibaredI}\color{nounibaredI}\textbackslash \color{nounibaredI}\textbackslash \color{black} \\
Ein neuer Absatz entsteht durch eine \\
leere Zeile. \\
\color{nounibaredI}\color{unibablueI}\textbackslash\color{unibablueI}end\color{black}\color{black}\{document\} \\

 \normalsize
\end{ttfamily}
\end{column}
\begin{column}{0.45\textwidth}
\begin{ttfamily}\textbf{\color{unibablueI}\textbackslash begin\color{black}\{Umgebung\}}\end{ttfamily}\\
Es beginnt eine neue Umgebung, hier das eigentliche Dokument.\\[5mm]

\begin{ttfamily}\textbf{\color{unibablueI}\textbackslash end\color{black}\{Umgebung\}}\end{ttfamily}\\
Die mit \begin{ttfamily}\textbf{\color{unibablueI}\textbackslash begin}\color{black}\{\}\end{ttfamily}
eingeleitete Umgebung ist hier zu Ende.\\
Jede geöffnete Umgebung muss auch \textbf{wieder geschlossen} werden!
\\[5mm]

\begin{ttfamily}\textbf{\color{nounibaredII}$\backslash\backslash$}\color{black}\end{ttfamily} Zeilenumbruch\\
\end{column}
\end{columns}
\end{frame}

%-------------------------------------------------------


\begin{frame}
\frametitle{Ein erstes Anwendungsbeispiel}
\framesubtitle{Pakete}
\begin{columns}
\begin{column}{0.55\textwidth}
\begin{ttfamily}\footnotesize
\color{nounibaredI}\color{nounibaredI}\textbackslash documentclass\color{black}\color{nounibagreenI}[a4paper, pdftex, ngerman]\color{black}\{article\} \\
\color{nounibaredI}\color{nounibaredI}\textbackslash usepackage\color{black}\color{nounibagreenI}[utf8]\color{black}\{inputenc\} \\
\color{nounibaredI}\color{nounibaredI}\textbackslash usepackage\color{black}\color{nounibagreenI}[T1]\color{black}\{fontenc\} \\
\color{nounibaredI}\color{unibablueI}\textbackslash\color{unibablueI}begin\color{black}\color{black}\{document\} \\
Das ist ein einfaches Minidokument \\
ohne Besonderheiten. Zeilenumbrüche \\
funktionieren immer automatisch! \\
Mehrere \\
Leerzeichen hintereinander werden  \\
zu einem zusammengefasst. \\
Getrennt wird auch automatisch.\color{nounibaredI}\color{nounibaredI}\textbackslash \color{nounibaredI}\textbackslash \color{black} \\
Mit zwei Backslashs beginnt eine neue \\
Zeile.\color{nounibaredI}\color{nounibaredI}\textbackslash \color{nounibaredI}\textbackslash \color{black} \\
Ein neuer Absatz entsteht durch eine \\
leere Zeile. \\
\color{nounibaredI}\color{unibablueI}\textbackslash\color{unibablueI}end\color{black}\color{black}\{document\} \\

 \normalsize
\end{ttfamily}
\end{column}
\begin{column}{0.45\textwidth}
\begin{ttfamily}\textbf{ngerman}\end{ttfamily}\\
Für Deutschland typische Formatierungen und (Trenn)-Regeln\\[5mm]

\begin{ttfamily}\textbf{inputenc}\end{ttfamily}\\
Definiert den Zeichen-\\satz, der verwendet werden soll. Es sollte immer
\begin{ttfamily}\textbf{UTF-8}\end{ttfamily} verwendet werden, weil er universal auf
allen Betriebssystemen läuft.\\[5mm]

\begin{ttfamily}\textbf{fontenc}\end{ttfamily}\\
z. B. Umlaute\\
\end{column}
\end{columns}
\end{frame}

%-------------------------------------------------------

\begin{frame}
\frametitle{Zeichenkodierungen}
\framesubtitle{Exkurs}
\begin{columns}
\begin{column}{0.6\textwidth}
\image{\textwidth}{image/utf8.png}{UTF-8 in TeXstudio}{img:utf8}

\end{column}
\begin{column}{0.4\textwidth}
\vspace*{2pt}\\
Wird ein Dokument geöffnet, wird automatisch der richtige Zeichensatz benutzt.
Beim Erstellen neuer Dokumente wird die Datei in dem Format gespeichert, das im
 Editor voreingestellt ist. In den TeXstudio-Einstelllungen muss derselbe Zeichensatz verwendet werden, der
auch im erstellten \LaTeX -Dokument verwendet wird.\\
\end{column}
\end{columns}
\vspace*{4pt}
\textbf{Bei Gruppenarbeiten muss jedes Mitglied zwingend \underline{UTF-8} im
Editor einstellen, sonst ist Ärger so gut wie vorprogrammiert!} (Kaputte
Umlaute, Kompilierungsfehler uvm., wenn es nicht nur Windows-User gibt.)
\end{frame}

%-------------------------------------------------------

\begin{frame}
\frametitle{Ein erstes Anwendungsbeispiel}
\framesubtitle{Pakete (Forts.)}
\begin{columns}
\begin{column}{0.55\textwidth}
\begin{ttfamily}\footnotesize
\color{nounibaredI}\color{nounibaredI}\textbackslash documentclass\color{black}\color{nounibagreenI}[a4paper, pdftex, ngerman]\color{black}\{article\} \\
\color{nounibaredI}\color{nounibaredI}\textbackslash usepackage\color{black}\color{nounibagreenI}[utf8]\color{black}\{inputenc\} \\
\color{nounibaredI}\color{nounibaredI}\textbackslash usepackage\color{black}\color{nounibagreenI}[T1]\color{black}\{fontenc\} \\
\color{nounibaredI}\color{unibablueI}\textbackslash\color{unibablueI}begin\color{black}\color{black}\{document\} \\
Das ist ein einfaches Minidokument \\
ohne Besonderheiten. Zeilenumbrüche \\
funktionieren immer automatisch! \\
Mehrere \\
Leerzeichen hintereinander werden  \\
zu einem zusammengefasst. \\
Getrennt wird auch automatisch.\color{nounibaredI}\color{nounibaredI}\textbackslash \color{nounibaredI}\textbackslash \color{black} \\
Mit zwei Backslashs beginnt eine neue \\
Zeile.\color{nounibaredI}\color{nounibaredI}\textbackslash \color{nounibaredI}\textbackslash \color{black} \\
Ein neuer Absatz entsteht durch eine \\
leere Zeile. \\
\color{nounibaredI}\color{unibablueI}\textbackslash\color{unibablueI}end\color{black}\color{black}\{document\} \\

 \normalsize
\end{ttfamily}
\end{column}
\begin{column}{0.45\textwidth}
\begin{ttfamily}\textbf{babel}\end{ttfamily}\\
Stellt sprachspezifische Infomationen (Formatierungen, Silbentrennung, Sonderzeichen) bereit.\\
\begin{itemize}
\item deutsch: \begin{ttfamily}ngerman\end{ttfamily}
\item englisch: \begin{ttfamily}english\end{ttfamily}
\end{itemize}
Benutzerdefinierte Trennungen kann man via {\begin{ttfamily} \color{unibaredI}\textbackslash - \end{ttfamily}} direkt angeben.\\

\begin{alertblock}{Achtung}
Die vorgestellten Packages sollten in jedem Dokument eingebunden werden!
\end{alertblock}
\end{column}
\end{columns}
\end{frame}



%-------------------------------------------------------

\begin{frame}
\frametitle{Abschnitte}
\framesubtitle{Kapitelmarken}
\begin{columns}
\begin{column}{0.5\textwidth}
\begin{ttfamily}\footnotesize
\color{nounibaredI}\color{nounibaredI}\textbackslash documentclass\color{black}\color{nounibagreenI}[a4paper, pdftex, 12pt, ngerman]\color{black}\{article\} \\
\color{nounibaredI}\color{nounibaredI}\textbackslash usepackage\color{black}\color{nounibagreenI}[utf8]\color{black}\{inputenc\} \\
\color{nounibaredI}\color{nounibaredI}\textbackslash usepackage\color{black}\color{nounibagreenI}[T1]\color{black}\{fontenc\} \\
\color{nounibaredI}\color{nounibaredI}\textbackslash usepackage\color{black}\{babel\} \\
\color{nounibaredI}\color{unibablueI}\textbackslash\color{unibablueI}begin\color{black}\color{black}\{document\} \\
\color{nounibaredI}\color{nounibaredI}\textbackslash tableofcontents\color{black} \\
\color{nounibaredI}\color{unibablueI}\textbackslash\color{unibablueI}section\color{black}\color{black}\{Kapitel 1\} \\
Hier kommt der erste Teil. \\
\color{nounibaredI}\color{unibablueI}\textbackslash\color{unibablueI}subsection\color{black}\color{black}\{Unterkapitel 1\} \\
Das erste Unterkapitel. \\
\color{nounibaredI}\color{unibablueI}\textbackslash\color{unibablueI}subsection\color{black}\color{black}\{Unterkapitel 2\} \\
Und noch ein Unterkapitel. \\
\color{nounibaredI}\color{unibablueI}\textbackslash\color{unibablueI}subsubsection\color{black}\color{black}\{Unterunterkapitel 1\} \\
Das ist ein Unterkapitel von einem Unterkapitel. \\
\color{nounibaredI}\color{unibablueI}\textbackslash\color{unibablueI}end\color{black}\color{black}\{document\} \\

\end{ttfamily}
\end{column}
\begin{column}{0.5\textwidth}

\begin{ttfamily}\color{nounibaredI}\textbackslash tableofcontents\color{black}\end{ttfamily}\\
Automatisches Inhaltsverzeichnis\\[3mm]
\begin{ttfamily}\color{nounibaredI}\textbackslash newpage\color{black}\end{ttfamily}\\
Seitenumbruch\\[3mm]
\begin{ttfamily}\color{unibablueI}\textbackslash section\color{black}\{Titel\}\end{ttfamily}\\
Ein neuer Abschnitt mit dem in \begin{ttfamily}\{\}\end{ttfamily} angegebenen Titel
beginnt.\\[3mm]
\begin{ttfamily}\color{unibablueI}\textbackslash subsection\color{black}\{Titel\}\end{ttfamily}\\
Ein Unterabschnitt.\\[3mm]
\begin{ttfamily}\color{unibablueI}\textbackslash subsubsection\color{black}\{Titel\}\end{ttfamily}\\
Noch eine Ebene darunter.\\
\end{column}
\end{columns}
\end{frame}

%-------------------------------------------------------

\begin{frame}
\frametitle{Abschnitte}
\framesubtitle{Kapitelmarken .PDF}
\begin{columns}
\begin{column}{0.5\textwidth}
\begin{ttfamily}\footnotesize
\color{nounibaredI}\color{nounibaredI}\textbackslash documentclass\color{black}\color{nounibagreenI}[a4paper, pdftex, 12pt, ngerman]\color{black}\{article\} \\
\color{nounibaredI}\color{nounibaredI}\textbackslash usepackage\color{black}\color{nounibagreenI}[utf8]\color{black}\{inputenc\} \\
\color{nounibaredI}\color{nounibaredI}\textbackslash usepackage\color{black}\color{nounibagreenI}[T1]\color{black}\{fontenc\} \\
\color{nounibaredI}\color{nounibaredI}\textbackslash usepackage\color{black}\{babel\} \\
\color{nounibaredI}\color{unibablueI}\textbackslash\color{unibablueI}begin\color{black}\color{black}\{document\} \\
\color{nounibaredI}\color{nounibaredI}\textbackslash tableofcontents\color{black} \\
\color{nounibaredI}\color{unibablueI}\textbackslash\color{unibablueI}section\color{black}\color{black}\{Kapitel 1\} \\
Hier kommt der erste Teil. \\
\color{nounibaredI}\color{unibablueI}\textbackslash\color{unibablueI}subsection\color{black}\color{black}\{Unterkapitel 1\} \\
Das erste Unterkapitel. \\
\color{nounibaredI}\color{unibablueI}\textbackslash\color{unibablueI}subsection\color{black}\color{black}\{Unterkapitel 2\} \\
Und noch ein Unterkapitel. \\
\color{nounibaredI}\color{unibablueI}\textbackslash\color{unibablueI}subsubsection\color{black}\color{black}\{Unterunterkapitel 1\} \\
Das ist ein Unterkapitel von einem Unterkapitel. \\
\color{nounibaredI}\color{unibablueI}\textbackslash\color{unibablueI}end\color{black}\color{black}\{document\} \\

\end{ttfamily}
\end{column}
\begin{column}{0.5\textwidth}
\image{\textwidth}{image/chapters.png}{Die Kapitel werden automatisch mitgez\"ahlt}{img:chapters}
\end{column}
\end{columns}
\end{frame}

%-------------------------------------------------------

\begin{frame}
\frametitle{Abschnitte}
\framesubtitle{Part \& Chapter}

Weitere Untergliederungsmöglichkeiten (neben \begin{ttfamily}\color{unibablueI}\textbackslash section\color{black}s\end{ttfamily}):
\begin{itemize}
\item  \begin{ttfamily}\color{unibablueI}\textbackslash part\color{black}\{\}\end{ttfamily}\\
Definiert größeren Teil und füllt eine ganze Seite.
\item \begin{ttfamily}\color{unibablueI}\textbackslash chapter\color{black}\{\}\end{ttfamily}\\
Gliedert in einzelne Kapitel, nur in {\ttfamily book} und {\ttfamily report}.
\end{itemize}


%\begin{columns}
%\begin{column}{0.5\textwidth}
%CODE
%\end{column}
%\begin{column}{0.5\textwidth}
%OUTPUT
%\end{column}
%\end{columns}
\end{frame}

%-------------------------------------------------------

\begin{frame}
\frametitle{Formatierungen}
\framesubtitle{Fett Kursiv Unterstrichen}
\begin{columns}
\begin{column}{0.5\textwidth}
\begin{ttfamily}\footnotesize
\color{nounibaredI}\color{nounibaredI}\textbackslash documentclass\color{black}\color{nounibagreenI}[a4paper, pdftex, 12pt, ngerman]\color{black}\{article\} \\
\color{nounibaredI}\color{nounibaredI}\textbackslash usepackage\color{black}\color{nounibagreenI}[utf8]\color{black}\{inputenc\} \\
\color{nounibaredI}\color{nounibaredI}\textbackslash usepackage\color{black}\color{nounibagreenI}[T1]\color{black}\{fontenc\} \\
\color{nounibaredI}\color{unibablueI}\textbackslash\color{unibablueI}begin\color{black}\color{black}\{document\} \\
Unter anderem folgende M"oglichkeiten:\color{nounibaredI}\color{nounibaredI}\textbackslash \color{nounibaredI}\textbackslash \color{black} \\
\color{nounibaredI}\color{nounibaredI}\textbackslash textbf\color{black}\{fetter\}\color{nounibaredI}\color{nounibaredI}\textbackslash \color{nounibaredI}\textbackslash \color{black} \\
\color{nounibaredI}\color{nounibaredI}\textbackslash textit\color{black}\{kursiver\}\color{nounibaredI}\color{nounibaredI}\textbackslash \color{nounibaredI}\textbackslash \color{black} \\
\color{nounibaredI}\color{nounibaredI}\textbackslash underline\color{black}\{unterstrichener\}\color{nounibaredI}\color{nounibaredI}\textbackslash \color{nounibaredI}\textbackslash \color{black} \\
\color{nounibaredI}\color{nounibaredI}\textbackslash underline\color{black}\{\color{nounibaredI}\color{nounibaredI}\textbackslash textbf\color{black}\{unterstrichen und fett\}\}\color{nounibaredI}\color{nounibaredI}\textbackslash \color{nounibaredI}\textbackslash \color{black} \\
\color{nounibaredI}\color{unibablueI}\textbackslash\color{unibablueI}end\color{black}\color{black}\{document\} \\

\end{ttfamily}
\end{column}
\begin{column}{0.5\textwidth}
Möglichkeiten:\\[3mm]
\textbf{fett}\\
\textit{kursiv}\\
\underline{unterstrichen}\\
\underline{\textbf{unterstrichen und fett}}
%\input{formats_pdf.tex}
\begin{block}{Textformatierungen}
\begin{ttfamily}$\color{nounibaredII}\backslash$\color{nounibaredII}textbf\color{black}\{Text\}\end{ttfamily}
fetter Text\\
\begin{ttfamily}$\color{nounibaredII}\backslash$\color{nounibaredII}textit\color{black}\{Text\}\end{ttfamily}
kursiver Text\\
\begin{ttfamily}$\color{nounibaredII}\backslash$\color{nounibaredII}underline\color{black}\{Text\}\end{ttfamily}
unterstrichen
\end{block}
\end{column}
\end{columns}
\end{frame}

%-------------------------------------------------------

\begin{frame}
\frametitle{Formatierungen}
\framesubtitle{Schriftgr"o\ss e}
\begin{columns}
\begin{column}{0.5\textwidth}
\begin{ttfamily}\footnotesize
\color{nounibaredI}\color{nounibaredI}\textbackslash tiny\color{black}~unlesbarer Text \\
\color{nounibaredI}\color{nounibaredI}\textbackslash scriptsize\color{black}~sehr kleiner Text \\
\color{nounibaredI}\color{nounibaredI}\textbackslash footnotesize\color{black}~Fussnotengröße \\
\color{nounibaredI}\color{nounibaredI}\textbackslash small\color{black}~klein \\
\color{nounibaredI}\color{nounibaredI}\textbackslash normalsize\color{black}~Standardgröße \\
\color{nounibaredI}\color{nounibaredI}\textbackslash large\color{black}~ größer \\
\color{nounibaredI}\color{nounibaredI}\textbackslash Large\color{black}~noch größer \\
\color{nounibaredI}\color{nounibaredI}\textbackslash LARGE\color{black}~sehr groß \\
\color{nounibaredI}\color{nounibaredI}\textbackslash huge\color{black}~riesig \\
\color{nounibaredI}\color{nounibaredI}\textbackslash Huge\color{black}~gigantisch \\

\end{ttfamily}
\begin{block}{Hinweis}
Nach Größenänderung muss mit \begin{ttfamily}$\color{nounibaredII}\backslash$\color{nounibaredII}normalsize\color{black}\end{ttfamily} wieder zurück gewechselt werden!\\
\end{block}
\end{column}
\begin{column}{0.5\textwidth}
\rm \tiny ~unlesbarer Text \\
\scriptsize ~sehr kleiner Text \\
\footnotesize ~Fu\ss notengr\"o\ss e\\
\small ~klein\\
\normalsize ~Standardgr\"o\ss e \\
\large ~gr\"o\ss er \\
\Large ~noch gr\"o\ss er \\
\LARGE ~sehr gro\ss \\
\huge  ~riesig \\
\Huge  ~gigantisch \\
\end{column}
\end{columns}
\begin{exampleblock}{Alternative}
\{\begin{ttfamily}$\color{nounibaredII}\backslash$\color{nounibaredII}large\color{black}\end{ttfamily} Größenänderung auf Klammer-Inhalt begrenzen\}
\end{exampleblock}
\vspace{-8mm}
\end{frame}

%TODO evtl Hinweisbox auf weitere Folie schieben und größen (Inkl hier fehlender small und Huge) nochmal auflisten

%-------------------------------------------------

\begin{frame}
\frametitle{Formatierungen}
\framesubtitle{Sonderzeichen}


\begin{columns}
\begin{column}{0.5\textwidth}
\begin{ttfamily}\footnotesize\color{nounibaredII}\textbackslash documentclass\color{nounibagreenI}[a4paper, pdftex, 12pt, ngerman]\color{black}\{article\}\\[3mm] 
$\color{nounibaredII}\backslash$\color{nounibaredII}usepackage\color{nounibagreenI}[utf8]\color{black}\{inputenc\}\\
$\color{nounibaredII}\backslash$\color{nounibaredII}usepackage\color{nounibagreenI}[T1]\color{black}\{fontenc\}\\
$\color{nounibaredII}\backslash$\color{nounibaredII}usepackage\color{black}\{babel\}\\
$\color{unibablueI}\backslash$\color{unibablueI}begin\color{black}\{document\}\\[3mm]
Dieser \color{gray}\%Kommentar wird nicht kompiliert.\\ \color{black}
Dies ist zu 100 \color{nounibaredII}\textbackslash \% \color{black} kein Kommentar.
\color{unibablueI}\textbackslash end\color{black}\{document\}
\end{ttfamily}
\end{column}
\begin{column}{0.5\textwidth}
\vspace*{3mm}\\
Sonderzeichen:
\begin{itemize}
\item Sind meist auch Steuerzeichen, z. B. wird das \% - Zeichen für Kommentare verwendet
\item Um sie trotzdem verwenden zu können, müssen sie mit dem '\color{nounibaredI}\textbackslash \color{black}' (Backslash) eingeführt bzw. escaped werden
\end{itemize}
\vspace*{5mm}
\image{.8\textwidth}{image/kommentar.png}{Beispieltext als .PDF}{img:kommentar}
\end{column}
\end{columns}

%TODO evtl kleines Bild mit Ergebnis




\end{frame}

%-------------------------------------------------

\begin{frame}
\frametitle{Formatierungen}
\framesubtitle{Sonderzeichen cont'd}
\begin{columns}
\begin{column}{0.5\textwidth}
\begin{ttfamily}\scriptsize\color{nounibaredII}\textbackslash documentclass\color{nounibagreenI}[a4paper, pdftex, 12pt, ngerman]\color{black}\{article\}\\ %[3mm] 
$\color{nounibaredII}\backslash$\color{nounibaredII}usepackage\color{nounibagreenI}[utf8]\color{black}\{inputenc\}\\
$\color{nounibaredII}\backslash$\color{nounibaredII}usepackage\color{nounibagreenI}[T1]\color{black}\{fontenc\}\\
$\color{nounibaredII}\backslash$\color{nounibaredII}usepackage\color{black}\{babel\}\\[2mm]
$\color{nounibaredII}\backslash$\color{nounibaredII}usepackage\color{black}\{eurosym\}\\
$\color{unibablueI}\backslash$\color{unibablueI}begin\color{black}\{document\}\\
\color{gray}\% Kommentar (nicht angezeigt)\\
$\color{nounibaredII}\backslash$\color{nounibaredII}textit\color{black}\{Einige
Sonderzeichen:\}\\
\color{nounibaredII}\textbackslash \% \textbackslash \$ \textbackslash \& \textbackslash \{ \textbackslash \}
\textbackslash \_ \textbackslash \# \textbackslash S \textbackslash copyright\\
\textbackslash slash \~ ~ \color{unibayellowI}\$\color{nounibaredII}$\color{nounibaredII}\backslash$backslash\color{unibayellowI}\$\color{nounibaredII}  ~\textbackslash euro \\

$\color{nounibaredII}\backslash$\color{nounibaredII}textit\color{black}\{Binde-\slash
Gedanken-\slash Trennstriche:\} \\
- -- --- \color{unibayellowI}\$\color{black}-\color{unibayellowI}\$\color{black} (letzteres mathematisches Minus) \\

$\color{nounibaredII}\backslash$\color{nounibaredII}textit\color{black}\{Anf"uhrungszeichen aus \begin{ttfamily}ngerman\end{ttfamily}:\} \\
\color{nounibaredII}\textbackslash glqq \textbackslash grqq \textbackslash flqq \textbackslash frqq\\
\color{unibablueI}\textbackslash end\color{black}\{document\}
\end{ttfamily}
\end{column}
\begin{column}{0.5\textwidth}
\textit{Einige Sonderzeichen:}    \\
\% \$ \& \{ \} \_ \# \S ~ \copyright \slash ~ \textbackslash  \euro \\

\textit{Binde-\slash Gedanken-\slash Trennstriche:} \\
- -- --- $-$ (letzteres mathematisches Minus) \\

\textit{Anführungszeichen aus ngerman:} \\
\glqq \grqq \flqq \frqq\\[5mm]
Für das \euro -Zeichen wird das Package \begin{ttfamily}eurosym\end{ttfamily}
benötigt.\\

\end{column}
\end{columns}
%\medskip
%\footnotesize Sonderzeichen \textbf{müssen} mit dem '\color{nounibaredI}\textbackslash \color{black}' eingeführt werden.
%Manchmal, z.B. in \"Uberschriften m\"ussen Umlaute des Pakets ngerman mit \grqq a \grqq o
%\grqq u und das \ss ~mit \color{nounibaredI} \textbackslash ss \color{black}gebildet werden, ansonsten reicht es das Package {\ttfamily babel} einzubinden.

%\bigskip 
\begin{block}{Hinweis}
\footnotesize Manchmal muss das \ss ~mit \color{nounibaredI} \textbackslash ss \color{black}gebildet werden, ansonsten reicht es das Package {\ttfamily babel} einzubinden.\\
Ein Leerzeichen (z. B. nach Sonderzeichen) wird durch ein \color{nounibaredI} $\sim$ \color{black} (Tilde) erzwungen.
\end{block}
\vspace{-8mm}

\end{frame}

%-------------------------------------------------

\begin{frame}
\frametitle{Formatierungen}
\framesubtitle{Sonderzeichen cont'd}

%\scriptsize

\begin{tabular}{|c|c|c|c|c|c|c|c|c|}
\hline
\ttfamily \color{nounibaredII}\textbackslash \% \color{black}&\color{nounibaredII}\ttfamily \textbackslash \$ \color{black}&\color{nounibaredII}\ttfamily  \textbackslash \&  \color{black}&\color{nounibaredII}\ttfamily \textbackslash \{ \color{black}&\color{nounibaredII}\ttfamily \textbackslash \} \color{black}&\color{nounibaredII}\ttfamily \textbackslash \_ \color{black}&\color{nounibaredII}\ttfamily \textbackslash \#  \color{black}&\color{nounibaredII}\ttfamily \textbackslash S \color{black}&\color{nounibaredII}\ttfamily \textbackslash copyright \color{black} \rmfamily \\ 
 \hline
\% & \$ & \& & \{ & \} & \_ & \# & \S & \copyright \\
 \hline
\end{tabular} 

\bigskip

\begin{tabular}{|c|c|c|c|c|}
\hline
\ttfamily \color{nounibaredII}\ttfamily \textbackslash slash \color{black} &\color{unibayellowI}\$\color{nounibaredII}\ttfamily $\color{nounibaredII}\backslash$backslash\color{unibayellowI}\$\color{nounibaredII}& \color{nounibaredII} \ttfamily \textbackslash textbackslash \color{black} & \color{nounibaredII} \ttfamily \textbackslash euro \color{black}& \ttfamily \~ ~ \rmfamily \\
 \hline
\slash & $\backslash$ & \textbackslash & \euro & ~  \\
 \hline
\end{tabular} 

\bigskip

\begin{tabular}{|c|c|c|c|c|c|c|c|}
\hline
\ttfamily \color{nounibaredII}\ttfamily \textbackslash glqq \color{black} & \color{nounibaredII} \ttfamily \textbackslash grqq \color{black} & \color{nounibaredII} \ttfamily \textbackslash flqq \color{black}&  \color{nounibaredII} \ttfamily \textbackslash frqq \color{black} & \ttfamily - & \ttfamily -- & \ttfamily --- & \color{unibayellowI}\$\color{nounibaredII}\ttfamily \color{nounibaredII}-\color{unibayellowI}\$\color{nounibaredII} \rmfamily \\
 \hline
\glqq & \grqq & \flqq & \frqq & - & --  & --- & - \\
 \hline
\end{tabular} 


%- -- --- $-$ (letzteres mathematisches Minus) \\
%
%\textit{Anführungszeichen aus ngerman:} \\
%\glqq \grqq \flqq \frqq\\[5mm]
%Für das \euro -Zeichen wird das Package \begin{ttfamily}eurosym\end{ttfamily}
%benötigt.\\
%
%
\bigskip
\begin{block}{Hinweis}
 Sonderzeichen \textbf{müssen} mit dem '\color{nounibaredI}\textbackslash \color{black}' eingeführt werden.

\end{block}
%
%
%%\vspace{-8mm}

\end{frame}

%-------------------------------------------------

%TODO required?
\begin{frame}
\frametitle{Formatierungen}
\framesubtitle{Textausrichtung}
\begin{columns}
\begin{column}{0.5\textwidth}
\begin{ttfamily}\footnotesize
%\color{unibablueI}\textbackslash \color{unibablueI}begin\color{black}\{document\}\\
%\vspace{3mm}
\color{unibablueI}\textbackslash \color{unibablueI}begin\color{black}\{flushleft\}\\
Nach links ausgerichteter Text.\\
\color{unibablueI}\textbackslash \color{unibablueI}end\color{black}\{flushleft\}\\
\vspace{2mm}
\color{unibablueI}\textbackslash \color{unibablueI}begin\color{black}\{center\}\\
Mittig ausgerichteter Text.\\
\color{unibablueI}\textbackslash \color{unibablueI}end\color{black}\{center\}\\
\vspace{2mm}
\color{unibablueI}\textbackslash \color{unibablueI}begin\color{black}\{flushright\}\\
Nach rechts ausgerichteter Text.\\
\color{unibablueI}\textbackslash \color{unibablueI}end\color{black}\{flushright\}\\
%\vspace{3mm}
%\color{unibablueI}\textbackslash end\color{black}\{document\}\\
\end{ttfamily}
\end{column}
\begin{column}{0.5\textwidth}
Umgebungen zur Textausrichung:\\
\begin{itemize}
\item flushleft	
\item center
\item flushright
\end{itemize}
\end{column}
\end{columns}
\begin{flushleft}
Links ausgerichteter Text.\\
\end{flushleft}
\begin{center}
Mittig ausgerichteter Text.\\
\end{center}
\begin{flushright}
Rechts ausgerichteter Text.\\
\end{flushright}
\end{frame}
