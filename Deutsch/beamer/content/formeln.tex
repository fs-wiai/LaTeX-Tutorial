\section{Formeln}
\begin{frame}
\frametitle{Formeln}
\framesubtitle{Mathematische Formeln einbinden}

\begin{exampleblock}{Neue Pakete in diesem Abschnitt}
\begin{multicols}{2}
\begin{itemize}
\item amsmath 
\item amsthm
\item amssymb
\item mathtools
\end{itemize}
\end{multicols}
\end{exampleblock}

\begin{block}{Neue Befehle in diesem Abschnitt}
\begin{multicols}{2}
\begin{itemize}
\item \color{nounibaredI}\textbackslash sqrt\color{black}\{\}
\item \color{nounibaredI}\textbackslash frac\color{black}\{\}\{\}
\item \color{nounibaredI}\textbackslash int\color{black}\_X
\item \color{nounibaredI}\textbackslash sum\color{black}\_\{\}
\item \color{nounibaredI}\textbackslash lim\color{black}\_\{\}
\item \color{nounibaredI}\textbackslash prod\color{black}
\item \color{nounibaredI}\textbackslash limits\color{black}\_\{\}
\item \color{nounibaredI}\textbackslash ldots\color{black}
\item \color{nounibaredI}\textbackslash cdot\color{black}
\item \color{nounibaredI}\_\color{black}
\item \color{nounibaredI}\^~\color{black}
\item und viele mehr ...
\end{itemize}
\end{multicols}
\end{block}

\end{frame}

%-------------------------------------------------------------------------------
\begin{frame}
\frametitle{Formeln}
\framesubtitle{Zerlegung eines Ausdrucks in seine Bestandteile}

\begin{columns}
	\begin{column}{.3\textwidth}
		{\huge $2 \sqrt{\frac{\pi ^2}{3}\cdot c_{2}}$}
	\end{column}
	
	\begin{column}{.7\textwidth}
		$\underbrace{
			\color{unibayellowI}\text{\$}
			\color{black}2
			\color{nounibaredI}\backslash \text{sqrt}
			\color{black}\{
			\color{nounibaredI}\backslash \text{frac}
			\color{black}\{
			\color{nounibaredI}\backslash \text{pi}\color{nounibaredI}
			~\hat{}~\color{black}2\}\{3\color{black}\}
			\color{nounibaredI}\backslash
			\color{nounibaredI}\text{cdot}~
			\color{black} \text{c}
			\color{nounibaredI}\_
			\color{black}2\}
			\color{unibayellowI}\text{\$}
		}$
		\color{black}
		
		Die Formel-Umgebung wird durch \color{unibayellowI}\$ \color{black} angefangen und beendet.
		
		\medskip
		
		$\underbrace{
			\color{nounibaredI}\backslash \text{sqrt}
			\color{black}\{
			\color{nounibaredI}\backslash \text{frac}
			\color{black}\{
			\color{nounibaredI}\backslash \text{pi}\color{nounibaredI}
			~\hat{}~\color{black}2\}\{3\color{black}\}
			\color{nounibaredI}\backslash
			\color{nounibaredI}\text{cdot}~
			\color{black} \text{c}
			\color{nounibaredI}\_
			\color{black}2\}
		}$
		\color{black}
		
		Die Wurzel.
		
		\medskip
		
		$\underbrace{
			\color{nounibaredI}\backslash \text{frac}
			\color{black}\{
			\color{nounibaredI}\backslash \text{pi}\color{nounibaredI}
			~\hat{}~\color{black}2\}\{3\color{black}\}
		}$
		
		Ein Bruch hat immer Z"ahler und Nenner.
	\end{column}
\end{columns}

%\bigskip
%
%\begin{block}{Tipp}
%Um Ausdrücke in einer eigenen Zeile hervorzuheben, kann man auch {\ttfamily\color{unibayellowI}\$\$\color{nounibaredI} \ldots \color{unibayellowI}\$\$} schreiben. 
%$$ Wie \sqrt[z.B.]{hier} $$
%zwischen dem Text.
%\end{block}
%\vspace{-8mm}

\end{frame}


%-------------------------------------------------------------------------------

\begin{frame}
\frametitle{Formeln}
\framesubtitle{\ldots ~in \LaTeX ~eine wahre Sch\"onheit!}
\begin{columns}
\begin{column}{.4\textwidth}
\flushright
$\int_0^\infty$
\end{column}
\begin{column}{.6\textwidth}
\flushleft
{\ttfamily\color{unibayellowI}\$\color{nounibaredI}\textbackslash\color{nounibaredI}int\_\color{black}0\color{nounibaredI}\textasciicircum \textbackslash infty\color{unibayellowI}\$}
\end{column}
\end{columns}
\begin{columns}
\begin{column}{.4\textwidth}
\flushright
$\sum_{i=1}^n$
\end{column}
\begin{column}{.6\textwidth}
\flushleft
{\ttfamily \color{unibayellowI}\$\color{nounibaredI}\textbackslash
\color{nounibaredI}sum\_\color{black}\{i=1\}\color{nounibaredI}\textasciicircum
\color{black}n\color{unibayellowI}\$}
\end{column}
\end{columns}

\begin{columns}
\begin{column}{.4\textwidth}
\flushright
$\lim_{n \rightarrow \infty}$
\end{column}
\begin{column}{.6\textwidth}
\flushleft
{\ttfamily \color{unibayellowI}\$\color{nounibaredI}\textbackslash
\color{nounibaredI}lim\_\color{black}\{n \color{nounibaredI}\textbackslash
\color{nounibaredI}rightarrow \color{nounibaredI}\textbackslash infty\color{black}\}\color{unibayellowI}\$}
\end{column}
\end{columns}

\begin{columns}
\begin{column}{.4\textwidth}
\flushright
$\prod\limits_{i=1}^{n+1}i = 1 \cdot 2 \cdot \ldots \cdot n \cdot (n+1)$
\end{column}
\begin{column}{.6\textwidth}
\flushleft
{\ttfamily \color{unibayellowI}\$%
\color{nounibaredI}\textbackslash\color{nounibaredI}prod\textbackslash  limits\_\color{black}\{i=1\}\color{nounibaredI}\^{}\color{black}\{n+1\} i = 1 \color{nounibaredI}\textbackslash \color{nounibaredI}cdot \color{black}2 \color{nounibaredI}\textbackslash \color{nounibaredI}cdot \color{nounibaredI}\textbackslash \color{nounibaredI}ldots \color{nounibaredI}\textbackslash\color{nounibaredI}cdot \color{black}n \color{nounibaredI}\textbackslash \color{nounibaredI}cdot \color{black}(n+1)\color{unibayellowI}\$}
\end{column}
\end{columns}
\bigskip
Die  American Mathematical Society hat einen n"utzlichen Guide f"ur das {\ttfamily amsmath}-Package\footnote{\url{ftp://ftp.ams.org/pub/tex/doc/amsmath/amsldoc.pdf}}.
\end{frame}

%-------------------------------------------------------------------------------
\begin{frame}
\frametitle{Formeln}
\framesubtitle{Hervorheben eines Ausdrucks}



Um Ausdrücke in einer eigenen Zeile abzusetzen, kann man sie auch in {\ttfamily\color{unibayellowI}\textbackslash [\color{nounibaredI} \ldots \color{unibayellowI}\textbackslash ]} schreiben. 
Ein Beispiel ist folgender Ausdruck:
\[ \sum\limits_{i=1}^{n+1} (n \cdot 2) - 5 \]
Nach ihm geht der weitere Text einfach weiter.

\medskip

Code:\\
{\ttfamily \color{unibayellowI}\textbackslash [	\color{nounibaredI}\textbackslash\color{nounibaredI}sum\textbackslash  limits\_\color{black}\{i=1\}\color{nounibaredI}\^{}\color{black}\{n+1\} (n \color{nounibaredI}\textbackslash \color{nounibaredI}cdot \color{black}2) - 5 \color{unibayellowI}\textbackslash ]}

\medskip

\begin{block}{Indizes und Potenzen}
Durch {\ttfamily \color{nounibaredI}\^{}\color{black}\{\} und \color{nounibaredI}\_\color{black}\{\} } werden Inhalte hoch- und tiefgestellt.

\begin{center}
{\ttfamily \color{unibayellowI}\$%
\color{black}x\color{nounibaredI}\_\color{black}\{i,j\}\color{black}~= i\color{nounibaredI}\^{}\color{black}\{j\}\color{unibayellowI}\$}

\medskip
 
$x_{i,j} = i^{j}$
\end{center}

\vspace{-2mm}

\end{block}
\vspace{-5mm}

\end{frame}



%-------------------------------------------------------------------------------



\begin{frame}
\frametitle{Formeln}
\framesubtitle{Ausgewählte nützliche Ausdrücke}
\begin{columns}
\begin{column}{.4\textwidth}
Mengensymbole:\\
$ |A \cup B| = |A|+|B| - |A \cap B| $
\end{column}
\begin{column}{.6\textwidth}
{\ttfamily \color{unibayellowI} \$\color{black} |A \color{nounibaredI}\textbackslash cup \color{black} B| = |A|+|B| - |A \color{nounibaredI}\textbackslash cap \color{black} B|\color{unibayellowI}\$ }
\end{column}
\end{columns}
\bigskip
\begin{columns}
\begin{column}{.4\textwidth}
Logische Verknüpfungen:\\
$ \land, \lor, \oplus, \lnot $
\end{column}
\begin{column}{.6\textwidth}
{\ttfamily \color{unibayellowI} \$ \color{nounibaredI}\textbackslash land\color{black}, \color{nounibaredI}\textbackslash lor\color{black}, \color{nounibaredI}\textbackslash oplus\color{black}, \color{nounibaredI}\textbackslash lnot \color{unibayellowI} \$ }
\end{column}
\end{columns}
\bigskip
\begin{columns}
\begin{column}{.4\textwidth}
Multiplikation/ Kreuzprodukt:\\
$ \cdot, \times $
\end{column}
\begin{column}{.6\textwidth}
{\ttfamily \color{unibayellowI} \$ \color{nounibaredI}\textbackslash cdot\color{black}, \color{nounibaredI}\textbackslash times \color{unibayellowI} \$ }
\end{column}
\end{columns}

\bigskip
\begin{columns}
\begin{column}{.4\textwidth}
Vergleiche:\\
$ <, \leq, =, \neq, \geq, > $
\end{column}
\begin{column}{.6\textwidth}
{\ttfamily \color{unibayellowI} \$ \color{black} <, \color{nounibaredI}\textbackslash leq\color{black}, =,  \color{nounibaredI}\textbackslash neq\color{black}, \color{nounibaredI}\textbackslash geq\color{black}, > \color{unibayellowI} \$ }
\end{column}
\end{columns}

%\bigskip
%\begin{columns}
%\begin{column}{.4\textwidth}
%Relation beschriften:\\
%$ x + y \stackrel{Kommutativgesetz}= y + x $
%\end{column}
%\begin{column}{.6\textwidth}
%{\ttfamily \color{unibayellowI} \$ \color{black} x + y \color{nounibaredI}\textbackslash stackrel\color{black}\{Kommutativgesetz\}= y + x \color{unibayellowI} \$ }
%\end{column}
%\end{columns}
\end{frame}

%-------------------------------------------------------------------------------

\begin{frame}
\frametitle{Formeln}
\framesubtitle{Ausgewählte nützliche Ausdrücke}
\begin{columns}
\begin{column}{.4\textwidth}
Klammern:\\
$ (x), [x], \lbrace x \rbrace, \lvert x \rvert $
\end{column}
\begin{column}{.6\textwidth}
{\ttfamily \color{unibayellowI} \$ \color{black} (x), [x], \color{nounibaredI}\textbackslash lbrace \color{black} x \color{nounibaredI}\textbackslash rbrace, \color{nounibaredI}\textbackslash lvert \color{black} x \color{nounibaredI}\textbackslash rvert \color{unibayellowI} \$ }
\end{column}
\end{columns}

\bigskip

\begin{columns}
\begin{column}{.4\textwidth}
Diverse Symbole:\\
$ \exists, \forall, \in, \notin, \infty $
\end{column}
\begin{column}{.6\textwidth}
{\ttfamily \color{unibayellowI} \$ \color{nounibaredI}\textbackslash exists\color{black}, \color{nounibaredI}\textbackslash forall\color{black}, \color{nounibaredI}\textbackslash in\color{black}, \color{nounibaredI}\textbackslash notin\color{black}, \color{nounibaredI}\textbackslash infty \color{unibayellowI} \$ }
\end{column}
\end{columns}

\bigskip

\begin{columns}
\begin{column}{.4\textwidth}
Griechische Buchstaben:\\
$ \varepsilon, \sigma, \Pi, \delta $
\end{column}
\begin{column}{.6\textwidth}
{\ttfamily \color{unibayellowI} \$ \color{nounibaredI}\textbackslash varepsilon\color{black}, \color{nounibaredI}\textbackslash sigma\color{black}, \color{nounibaredI}\textbackslash Pi\color{black}, \color{nounibaredI}\textbackslash delta \color{unibayellowI} \$ }
\end{column}
\end{columns}

\bigskip
%
%\begin{columns}
%\begin{column}{.4\textwidth}
%Matrizen:\\
%$ \begin{pmatrix}
%	0 			& \alpha \\
%	\beta 	& 1
%\end{pmatrix} $\\
%Andere Klammerungen mit bmatrix, Bmatrix, vmatrix und Vmatrix
%\end{column}
%\begin{column}{.6\textwidth}
%\begin{ttfamily}
%\begin{tabbing}
%xx\=xx\=xxxxxx\=\kill
%\color{unibayellowI} \$\\
%\>\color{unibablueI}\textbackslash begin\color{black}\{pmatrix\}\\
%\>\>0\>\& \color{nounibaredI}\textbackslash alpha \textbackslash \textbackslash \\
%\>\>\color{nounibaredI}\textbackslash beta \color{black}\> \& 1\\
%\>\color{unibablueI}\textbackslash end\color{black}\{pmatrix\}\\
%\color{unibayellowI} \$\\
%\end{tabbing}
%\end{ttfamily}
%\end{column}
%\end{columns}
%
%\vspace{-8mm}

\medskip

\hspace{-4mm} Mitwachsende Klammern:\\ %TODO besserer name? Größe-Anpassende Klammern?
\medskip
\begin{columns}
\begin{column}{.2\textwidth}
\centering
$ \left( \frac{1}{2} \right) $
\end{column}
\begin{column}{.8\textwidth}
{\ttfamily \color{unibayellowI} \$ \color{nounibaredI}\textbackslash left\color{black}( \color{nounibaredI}\textbackslash frac\color{black}\{1\}\{2\} \color{nounibaredI}\textbackslash right\color{black} ) \color{unibayellowI} \$ }
\end{column}
\end{columns}

\medskip

\begin{columns}
\begin{column}{.2\textwidth}
\centering
$ \left\lbrace  \frac{1}{2} \right\rbrace $
\end{column}
\begin{column}{.8\textwidth}
{\ttfamily \color{unibayellowI} \$ \color{nounibaredI}\textbackslash left\textbackslash lbrace  \textbackslash frac\color{black}\{1\}\{2\} \color{nounibaredI}\textbackslash right\textbackslash rbrace \color{unibayellowI} \$ }
\end{column}
\end{columns}

\medskip


%\begin{columns}
%\begin{column}{.2\textwidth}
%\centering
%$ \left . \frac{1}{2} \right| $
%\end{column}
%\begin{column}{.8\textwidth}
%{\ttfamily \color{unibayellowI} \$ \color{nounibaredI}\textbackslash left \color{black}. \color{nounibaredI}\textbackslash frac\color{black}\{1\}\{2\} \color{nounibaredI}\textbackslash right\color{black} | \color{unibayellowI} \$ }
%\end{column}
%\end{columns}
%\bigskip



\begin{columns}
\begin{column}{.2\textwidth}
\centering
$ \left[ \frac{1}{2} \right] $
\end{column}
\begin{column}{.8\textwidth}
{\ttfamily \color{unibayellowI} \$ \color{nounibaredI}\textbackslash left\color{black}[ \color{nounibaredI}\textbackslash frac\color{black}\{1\}\{2\} \color{nounibaredI}\textbackslash right\color{black}] \color{unibayellowI} \$ }
\end{column}
\end{columns}




\end{frame}

%-------------------------------------------------------------------------------

\begin{frame}
\frametitle{Formeln}
\framesubtitle{Pfeile}

\bigskip

%Pfeile:\\

\begin{columns}
\begin{column}{.3\textwidth}
\centering
$ \rightarrow $
\end{column}
\begin{column}{.7\textwidth}
{\ttfamily \color{unibayellowI} \$ \color{nounibaredI}\textbackslash rightarrow \color{unibayellowI} \$ }
\end{column}
\end{columns}

\bigskip

\begin{columns}
\begin{column}{.3\textwidth}
\centering
$ \leftarrow $
\end{column}
\begin{column}{.7\textwidth}
{\ttfamily \color{unibayellowI} \$ \color{nounibaredI}\textbackslash leftarrow \color{unibayellowI} \$ }
\end{column}
\end{columns}

\bigskip

\begin{columns}
\begin{column}{.3\textwidth}
\centering
$ \Rightarrow $
\end{column}
\begin{column}{.7\textwidth}
{\ttfamily \color{unibayellowI} \$ \color{nounibaredI}\textbackslash Rightarrow \color{unibayellowI} \$ }
\end{column}
\end{columns}

\bigskip

\begin{columns}
\begin{column}{.3\textwidth}
\centering
$ \leftrightarrow $
\end{column}
\begin{column}{.7\textwidth}
{\ttfamily \color{unibayellowI} \$ \color{nounibaredI}\textbackslash leftrightarrow \color{unibayellowI} \$ }
\end{column}
\end{columns}

\bigskip

\begin{exampleblock}{Tipp}
TeXstudio bietet eine graphische Auswahl für sehr viele mathematische Symbole (und mehr \smiley).
\end{exampleblock}
%\vspace{-8mm}

\end{frame}

%-------------------------------------------------------------------------------

%\begin{frame}
%\frametitle{Formeln}
%\framesubtitle{Ausgewählte nützliche Ausdrücke}
%Align (Nicht innerhalb der Mathumgebung: {\ttfamily kein \color{unibayellowI} \$ ... \$ }setzen!!!):\\
%\medskip
%\begin{ttfamily}
%\begin{tabbing}
%xx\=xxxx\=xxxx\kill
%\color{unibablueI}\textbackslash begin\color{black}\{align\}\\
%\>i\textasciicircum 2\> =\& i \color{nounibaredI}\textbackslash cdot \color{black} i \&= \color{nounibaredI}\textbackslash sqrt\color{black}\{- 1\} \color{nounibaredI}\textbackslash cdot \textbackslash sqrt\color{black}\{-1\} \color{nounibaredI}\textbackslash \textbackslash \\
%\>\>=\& \color{nounibaredI}\textbackslash sqrt\color{black}\{(-1) \color{nounibaredI}\textbackslash cdot \color{black} (-1)\} \&= \color{nounibaredI}\textbackslash sqrt\color{black}\{1\} \color{nounibaredI}\textbackslash \textbackslash \\
%\>\>=\& 1 \&\color{nounibaredI}\textbackslash neq \color{black} -1\\
%\color{unibablueI}\textbackslash end\color{black}\{align\}\\
%\end{tabbing}
%\end{ttfamily}
%\begin{align}
%i^2 =& i \cdot i &= \sqrt{- 1} \cdot \sqrt{-1} \\
%=& \sqrt{(-1) \cdot (-1)} &= \sqrt{1} \\
%=& 1 &\neq -1
%\end{align}
%\end{frame}

%-------------------------------------------------------------------------------

