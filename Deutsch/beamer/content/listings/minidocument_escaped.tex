\color{nounibaredI}\color{nounibaredI}\textbackslash documentclass\color{black}\color{nounibagreenI}[a4paper, pdftex, 12pt, ngerman]\color{black}\{article\} \\
\color{nounibaredI}\color{nounibaredI}\textbackslash usepackage\color{black}\color{nounibagreenI}[utf8]\color{black}\{inputenc\} \\
\color{nounibaredI}\color{nounibaredI}\textbackslash usepackage\color{black}\color{nounibagreenI}[T1]\color{black}\{fontenc\} \\
\color{nounibaredI}\color{nounibaredI}\textbackslash usepackage\color{black}\{babel\} \\
\color{nounibaredI}\color{unibablueI}\textbackslash\color{unibablueI}begin\color{black}\color{black}\{document\} \\
Das ist ein einfaches Minidokument \\
ohne Besonderheiten. Zeilenumbrüche \\
funktionieren immer automatisch! \\
Mehrere \\
Leerzeichen hintereinander werden  \\
zu      einem zusammengefasst. \\
Getrennt wird auch automatisch.\color{nounibaredI}\color{nounibaredI}\textbackslash \color{nounibaredI}\textbackslash \color{black} \\
Mit zwei Backslashs beginnt eine neue \\
Zeile.\color{nounibaredI}\color{nounibaredI}\textbackslash \color{nounibaredI}\textbackslash \color{black} \\
Ein neuer Absatz entsteht durch eine \\
leere Zeile. \\
\color{nounibaredI}\color{unibablueI}\textbackslash\color{unibablueI}end\color{black}\color{black}\{document\} \\
