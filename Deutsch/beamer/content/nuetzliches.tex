

\subsection*{Nützliches}

\begin{frame}
\frametitle{Nützliches}
\framesubtitle{LaTeX-Werkzeuge}
\begin{itemize}
  \item \textbf{Symbolliste}\footnote{\url{http://tug.ctan.org/info/symbols/comprehensive/symbols-a4.pdf}} -- Liste aller \LaTeX -Symbole \\
  \item \textbf{Detexify}\footnote{\url{http://detexify.kirelabs.org/classify.html}} -- \LaTeX -Symbolsuche\\
  \item \textbf{Tables Generator}\footnote{\url{http://www.tablesgenerator.com/}} -- Online-Generator für \LaTeX -Tabellen  \\
  \item \textbf{Wikibooks}\footnote{\url{https://en.wikibooks.org/wiki/LaTeX}} -- Dokumentation für viele \LaTeX -Anwendungsfälle \\
  \item \textbf{Overleaf}\footnote{\url{https://www.overleaf.com/}} -- Tutorials zu verschiedenen Themen \\
\end{itemize}
\end{frame}

%----------------------------------------------------------------------

\begin{frame}
\frametitle{Nützliches}
\framesubtitle{Pakete}
\begin{itemize}
  \item \textbf{paralist}\footnote{\url{https://www.ctan.org/pkg/paralist}} -- Aufzählungen ohne unnötige Abstände \\
  \item \textbf{parskip}\footnote{\url{https://www.ctan.org/pkg/parskip}} -- kleine Abstände zwischen Absätzen \\
  \item \textbf{booktabs}\footnote{\url{https://www.ctan.org/pkg/booktabs}} -- typographisch schöne Tabellen \\
  \item \textbf{coffee}\footnote{\url{http://www.hanno-rein.de/downloads/coffee.pdf}} -- falls man Kaffeeflecken auf dem Dokument braucht \\
  \item \textbf{ltablex}\footnote{\url{https://www.ctan.org/pkg/ltablex}} -- vereint gleich zwei nützliche Tabellentools \\
  \item \textbf{minted}\footnote{\url{https://www.ctan.org/pkg/minted}} -- Syntax-Highlighting für Quelltext (benötigt Python) \\
\end{itemize}
\end{frame}

%----------------------------------------------------------------------

\begin{frame}
\frametitle{Nützliches}
\framesubtitle{Pakete}
\begin{itemize}
  \item \textbf{forest}\footnote{\url{https://www.ctan.org/pkg/forest}} -- zeichnet (Binär-)Bäume \\
  \item \textbf{tikz}\footnote{\url{https://www.ctan.org/pkg/pgf}} -- \glqq{}TikZ ist kein Zeichenprogramm\grqq \\
  \item \textbf{tcolorbox}\footnote{\url{https://www.ctan.org/pkg/tcolorbox}} -- Boxen für Beispiele \\
  \item \textbf{pdfpages}\footnote{\url{https://www.ctan.org/pkg/pdfpages}} -- Einbinden von PDF-Dateien \\
  \item \textbf{subcaption}\footnote{\url{https://www.ctan.org/pkg/subcaption}} -- Bildunterschriften auch in Subfigures \\
  \item \textbf{phfnote}\footnote{\url{https://www.ctan.org/pkg/phfnote}} -- Kompaktes Layout für Mitschriften und Abgaben \\
\end{itemize}
\end{frame}

%----------------------------------------------------------------------

\begin{frame}
\frametitle{Nützliches}
\framesubtitle{Pakete}
\begin{itemize}
  \item \textbf{todonotes}\footnote{\url{https://www.ctan.org/pkg/todonotes}} -- ToDo-Markierungen und Table of ToDos \\
  \item \textbf{cmbright}\footnote{\url{https://www.ctan.org/pkg/cmbright}} -- Serifenlose Schriften für \LaTeX \\
  \item \textbf{colortbl}\footnote{\url{https://www.ctan.org/pkg/colortbl}} -- Farbige \LaTeX -Tabellen \\
  \item \textbf{bbcard}\footnote{\url{https://www.ctan.org/pkg/bbcard}} -- Bullshit-Bingo-Karten \\
\end{itemize}
\end{frame}

%----------------------------------------------------------------------

\begin{frame}
\frametitle{Nützliches}
\framesubtitle{Eigene Befehle}
\begin{columns}
\hspace*{4.7mm}
\begin{column}{0.5\textwidth}
\textbf{Definition:}\\
\end{column}
\begin{column}{0.5\textwidth}
\textbf{Benutzung:}\\
\end{column}
\end{columns}
\bigskip
\begin{columns}
\hspace*{4.7mm}
\begin{column}{0.5\textwidth}
\begin{ttfamily}{\normalsize
\color{nounibaredI}\textbackslash newcommand\color{black}\{\textbackslash vektor\}[2]\{\\
\color{unibablueI}\textbackslash begin\color{black}\{pmatrix\}\\
\color{unibayellowI}\# 1 \color{nounibaredI}\textbackslash \textbackslash\\
\color{unibayellowI} \# 2\\
\color{unibablueI}\textbackslash end\color{black}\{pmatrix\}\\
\}\\
}
\end{ttfamily}
\end{column}
\begin{column}{0.5\textwidth}
\begin{ttfamily}{\normalsize
\color{unibayellowI}\$ \color{nounibaredI}\textbackslash vektor\color{black}\{3\}\{-2\} \color{unibayellowI}\$ \\}
\end{ttfamily}
\medskip
$
\begin{pmatrix}
3 \\ -2
\end{pmatrix}
$
\end{column}
\end{columns}
\bigskip
Weitere Infos zu vielen \LaTeX -Paketen findet ihr bei Wikibooks zu \LaTeX\footnote{\url{http://en.wikibooks.org/wiki/LaTeX}}.\\
\end{frame}

