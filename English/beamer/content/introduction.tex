\section{Intro} 

\subsection{Anwendungsbereiche, Sinn \& Zweck}
\begin{frame}[t]
\frametitle{Einf\"uhrung}
\framesubtitle{Sinn -- Unsinn -- Wahnsinn}
\bigskip
\bigskip
\bigskip

\begin{columns}[t]
\begin{column}{.3\textwidth}
\textbf{Sinnvoll}\\[3mm]
\begin{itemize}
\item Artikel
\item Bücher
\item wissenschaftliche Arbeiten
\item Bewerbungen
\end{itemize}
\end{column}
\begin{column}{.30\textwidth}
\textbf{Unsinn}\\[3mm]
\begin{itemize}
\item private Briefe
\item Geburtstags-einladungen
\item Getr"ankekarten
\end{itemize}
\end{column}
\begin{column}{.3\textwidth}
\textbf{Wahnsinn}\\[3mm]
\begin{itemize}
\item Einkaufszettel
\item Brainstorming
\item \ldots 
\end{itemize}
\end{column}
\end{columns}
\end{frame}

%-------------------------------------------------------------------------------

\begin{frame} 
\frametitle{Vom Code zum Dokument}
\framesubtitle{Kein WYSIWYG} 
\begin{columns}
\begin{column}{.7\textwidth}
\image{\textwidth}{image/worddoc.jpg}{\textbf{W}hat \textbf{Y}ou \textbf{S}ee \textbf{I}s \textbf{W}hat \textbf{Y}ou
\textbf{G}et}{img:worddoc}
\end{column}
\begin{column}{.3\textwidth}
\image{\textwidth}{image/codescreen.png}{\textbf{W}hat \textbf{W}ill \textbf{I} \textbf{G}et?}{img:codescreen}
%% Compile Animation
\end{column}
\end{columns}
\end{frame}

%-------------------------------------------------------------------------------

\begin{frame}
\frametitle{Einf\"uhrung}
\framesubtitle{Vorgehensweise}
\begin{columns}[onlytextwidth]
\begin{column}{0.40\textwidth}
\image{.8\textwidth}{image/codescreen.png}{Textdatei mit \LaTeX ~-Code}{img:code}
\end{column}
\begin{column}{0.25\textwidth}
\image{.8\textwidth}{image/miktex.jpg}{Compiler (z.B. MikTeX)}{img:miktex}
\end{column}
\begin{column}{0.25\textwidth}
\image{.6\textwidth}{image/pdflogo.png}{sch\"ones, lesbares und druckbares Dokument}{img:pdf}
\end{column}
\end{columns}
\end{frame}


%-------------------------------------------------------------------------------

\subsection{Vorteile \& Nachteile}
\begin{frame}
\frametitle{Einf\"uhrung}
\framesubtitle{Vorteile \& Nachteile}
Vorteile
\begin{itemize}
\item  dynamische Verzeichnisse und Referenzen
\item  automatische Layouts
\item  einfaches verteiltes Arbeiten möglich
\end{itemize}
Nachteile
\begin{itemize}
\item  Was kommt später raus?
\item  viele, zum Teil komplexe Befehle
\end{itemize}
\end{frame}

%-------------------------------------------------------------------------------

\subsection{\LaTeX --Compiler}
\begin{frame}
\frametitle{Einf\"uhrung}
\framesubtitle{\LaTeX - Compiler}
\begin{columns}[t]
\begin{column}{.4\textwidth}
\textbf{Software unter Windows:}\\
\begin{itemize}
  \item MikTex (http://www.miktex.org)\\
   2 Varianten: Basic oder  Complete
  \item ProTeXt (http://www.tug.org/protext)\\
enthält MikTex, TeXnicCenter und Ghostscript – einfache Installation\\
\end{itemize}
\end{column}
\begin{column}{.6\textwidth}
\textbf{Software unter *nix:}
\begin{itemize}
  \item TeXLive\\
Pakete unter Ubuntu: {\ttfamily texlive-full} ist das Meta-Paket mit allen
ben\"otigten Paketen. Enthält auch Folgende:
\begin{itemize}
  \item {\ttfamily texlive-base
  \item texlive-lang-german}
\end{itemize}
Installation: {\ttfamily sudo apt-get install texlive-full}
\item MacOS: MacTeX (http://www.tug.org/mactex/2009)\\
\end{itemize}
\end{column}
\end{columns}
\end{frame}

%-------------------------------------------------------------------------------

\subsection{Freie Editoren}

\subsubsection{*nix}
\begin{frame}
\frametitle{Einf\"uhrung}
\framesubtitle{Freie Editoren -- Linus \& MacOS }
\begin{itemize}
 \item Kile\footnote{http://kile.sourceforge.net/}\\KDE-Programm, auch unter Gnome\slash Unity etc.
 verwendbar. Installation auf Debiansystemen mit {\ttfamily sudo apt-get
 install kile}.
  \item Vim \LaTeX -suite (Plugin)\footnote{http://vim-latex.sourceforge.net/}\\
  Ein Traum f"ur Vim-User.
  \item TexShop (MacOS)\footnote{http://pages.uoregon.edu/koch/texshop/}
\end{itemize}
\end{frame}

%-------------------------------------------------------------------------------

\subsubsection{Windows}
\begin{frame}
\frametitle{Einf\"uhrung}
\framesubtitle{Freie Editoren -- Windows}
\begin{itemize}
\item TeXnicCenter\footnote{http://www.texniccenter.org/}
  %\item \ldots
\end{itemize}
\end{frame}

%-------------------------------------------------------------------------------

\subsubsection{Cross-Platform}
\begin{frame}
\frametitle{Einf\"uhrung}
\framesubtitle{Freie Editoren -- Cross-Platform}
\begin{itemize}
  \item TeXMaker\footnote{http://www.xm1math.net/texmaker}\\
   Sehr solide, verwenden wir hier im Tutorium.
  \item TeXstudio\footnote{http://sourceforge.net/projects/texstudio/?source=dlp}\\
  "Ahnlich wie der TeXMaker, allerdings etwas m"achtiger.
  \item TeXlipse\footnote{http://texlipse.sourceforge.net/}\\ F"ur fortgeschrittene User, Plugin f"ur
  Eclipse. Gute IDE-Unterst\"utzung, Code-Completion, Autobuilds, Versionsverwaltung etc.
\end{itemize}
\end{frame}

%-------------------------------------------------------------------------------

\begin{frame}
\frametitle{TeXmaker}
\framesubtitle{\"Uberblick}
\image{\textwidth}{image/texmaker_overview.png}{Das Standardfenster des Texmaker}{img:texmaker1}

\end{frame}

%-------------------------------------------------------------------------------

\begin{frame}
\frametitle{TeXmaker}
\framesubtitle{Synctex}
\image{\textwidth}{image/synctex.png}{Synctex}{img:synctex}

\end{frame}

\begin{frame}
\frametitle{Mein erstes \LaTeX~-Dokument}
\begin{block}{Neue Befehle:}
\begin{itemize}
\item \begin{ttfamily}\color{nounibaredII}\textbackslash documentclass\color{nounibagreenI}\color{black}\{article\}\end{ttfamily}
\item \begin{ttfamily}\color{unibablueI}\textbackslash begin\color{black}\{document\}\end{ttfamily}
\item \begin{ttfamily}\color{unibablueI}\textbackslash end\color{black}\{document\}\end{ttfamily}
\end{itemize}
\end{block}
Das ist alles was man f\"ur ein \LaTeX -Dokument braucht. Und das probieren wir jetzt aus!

\end{frame}
