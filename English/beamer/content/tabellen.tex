\section{Tables}

\begin{frame}
\frametitle{Tables}
\framesubtitle{Insertion Of Tables}

\begin{exampleblock}{New packages in this section}
\begin{itemize}
\item longtable
\end{itemize}
\end{exampleblock}

\begin{block}{New commands in this section}
\begin{itemize}
\item \color{unibablueI}\textbackslash begin\color{black}\{tabular\} \dots
~\color{unibablueI}\textbackslash end\color{black}\{tabular\}
\item \color{unibablueI}\textbackslash begin\color{black}\{table\} \dots
~\color{unibablueI}\textbackslash end\color{black}\{table\}
\item \color{unibablueI}\textbackslash begin\color{black}\{longtable\} \dots
~\color{unibablueI}\textbackslash end\color{black}\{longtable\}
\item \color{unibablueI}\textbackslash begin\color{black}\{tabbing\} \dots
~\color{unibablueI}\textbackslash end\color{black}\{tabbing\}
\item \color{nounibaredI}$|$\color{black}
\item \color{nounibaredI}\& \color{black}
\item \color{nounibaredI}\textbackslash hline\color{black}
\item \color{nounibaredI}\textbackslash multicolumn\color{black}\{\}\{\}\{\}
\end{itemize}
\end{block}

\end{frame}

\begin{frame}
\frametitle{Tables}
\framesubtitle{``table'' \& \glqq tabular\grqq}
\textbf{Structure:}\\[2mm]
\color{unibablueI}\begin{ttfamily}\textbackslash begin\color{black}\{table\}\color{nounibagreenI}[position]\color{black}\\
\color{unibablueI}\textbackslash begin\color{black}\{tabular\}\{\textit{Definition of columns}\}\\
\textit{Table content}\\
\color{unibablueI}\textbackslash end\color{black}\{tabular\}\\
\color{nounibaredI}\textbackslash caption\color{black}\{caption\}\\
\color{nounibaredI}\textbackslash label\color{black}\{tab:bsptab1\}\\
\color{unibablueI}\textbackslash end\color{black}\{table\}\\
~\\
\end{ttfamily}

\begin{block}{Reminder: positioning in most \LaTeX -- environments}
\color{nounibagreenI}[h]\color{black}~or \color{nounibagreenI}[H]\color{black}~= At this very position\\
\color{nounibagreenI}[t]\color{black}~= On top of the page\\ 
\color{nounibagreenI}[b]\color{black}~= On bottom of the page\\ 
\color{nounibagreenI}[p]\color{black}~= Positioning on an own page
\end{block}
\end{frame}


\begin{frame}
\frametitle{Tables}
\framesubtitle{Definition Of Columns}
Here you can define how the columns should be aligned
and how the vertical lines should be set:\\[3mm]
\begin{tabbing}[H]p{column width}xxx\=\kill
\textbf{Commands:}\\
l \>= left-justified\\
c \>= centered\\
r \>= right-justified\\
p\{column width\} \>= a left-justified column with defined width\\
\color{nounibaredI}$|$\color{black} \>= sets a vertical line
at this position\\
\end{tabbing}
\end{frame}

\begin{frame}
\frametitle{Tables}
\framesubtitle{View From Inside}
\begin{ttfamily}
\color{nounibaredI}\color{unibablueI}\textbackslash\color{unibablueI}begin\color{black}\color{black}\{tabular\}\{c\color{nounibaredI}|\color{black}p\{40mm\}\color{nounibaredI}|\color{black}lr\color{nounibaredI}|\color{black}c\} \\
\color{nounibaredI}\color{nounibaredI}\textbackslash multicolumn\color{black}\{5\}\{c\}\{E-Sports Championship Franconia\} \color{nounibaredI}\color{nounibaredI}\textbackslash \color{nounibaredI}\textbackslash \color{black} \\
\color{nounibaredI}\color{nounibaredI}\textbackslash hline\color{black} \\
\color{nounibaredI}\color{nounibaredI}\textbackslash hline\color{black} \\
Number \color{nounibaredI}\&  \color{black}Place \color{nounibaredI}\&  \color{black}Player 1 \color{nounibaredI}\&  \color{black}Player 2 \color{nounibaredI}\&  \color{black}Result \color{nounibaredI}\color{nounibaredI}\textbackslash \color{nounibaredI}\textbackslash \color{black} \\
\color{nounibaredI}\color{nounibaredI}\textbackslash hline\color{black} \\
1 \color{nounibaredI}\&  \color{black}Nürnberg \color{nounibaredI}\&  \color{black}Wolf \color{nounibaredI}\&  \color{black}Lamm \color{nounibaredI}\&  \color{black}23:10 \color{nounibaredI}\color{nounibaredI}\textbackslash \color{nounibaredI}\textbackslash \color{black} \\
\color{nounibaredI}\color{nounibaredI}\textbackslash hline\color{black} \\
2 \color{nounibaredI}\&  \color{black}Bamberg \color{nounibaredI}\&  \color{black}Meyer \color{nounibaredI}\&  \color{black}Beyer \color{nounibaredI}\color{nounibaredI}\textbackslash \color{nounibaredI}\textbackslash \color{black} \\
\color{nounibaredI}\color{nounibaredI}\textbackslash hline\color{black} \\
3 \color{nounibaredI}\&  \color{black}Zirndorf \color{nounibaredI}\&  \color{black}Brandst. \color{nounibaredI}\&  \color{black}Brauer \color{nounibaredI}\&  \color{black}21:21\color{nounibaredI}\color{nounibaredI}\textbackslash \color{nounibaredI}\textbackslash \color{black} \\
\color{nounibaredI}\color{nounibaredI}\textbackslash hline\color{black} \\
\color{nounibaredI}\color{unibablueI}\textbackslash\color{unibablueI}end\color{black}\color{black}\{tabular\} \\

\end{ttfamily}
\end{frame}

\begin{frame}
\frametitle{Tables}
\framesubtitle{Table Content}
Here you fill the defined columns with content.\\[3mm]
\begin{tabbing}[H]p{column width}xxx\=\kill
\textbf{Commands:}\\
\color{nounibaredI}\&\color{black} \>= horizontal separation of rows\\
\color{nounibaredI}\textbackslash \textbackslash \color{black} \>=  new line\\
\color{nounibaredI}\textbackslash hline\color{black} \>= sets a horizontal line\\[2mm]
\color{nounibaredI}\textbackslash multicolumn\color{black}\{column number\}\{column alignment\}\{text\}\\[2mm]
\>= Combines as many columns as you like.\\
\end{tabbing}
\end{frame}

\begin{frame}[t]

\frametitle{Tables}
\framesubtitle{Example Tabular}
\begin{footnotesize}
\begin{ttfamily}
\color{nounibaredI}\color{unibablueI}\textbackslash\color{unibablueI}begin\color{black}\color{black}\{tabular\}\{c\color{nounibaredI}|\color{black}p\{40mm\}\color{nounibaredI}|\color{black}lr\color{nounibaredI}|\color{black}c\} \\
\color{nounibaredI}\color{nounibaredI}\textbackslash multicolumn\color{black}\{5\}\{c\}\{E-Sports Championship Franconia\} \color{nounibaredI}\color{nounibaredI}\textbackslash \color{nounibaredI}\textbackslash \color{black} \\
\color{nounibaredI}\color{nounibaredI}\textbackslash hline\color{black} \\
\color{nounibaredI}\color{nounibaredI}\textbackslash hline\color{black} \\
Number \color{nounibaredI}\&  \color{black}Place \color{nounibaredI}\&  \color{black}Player 1 \color{nounibaredI}\&  \color{black}Player 2 \color{nounibaredI}\&  \color{black}Result \color{nounibaredI}\color{nounibaredI}\textbackslash \color{nounibaredI}\textbackslash \color{black} \\
\color{nounibaredI}\color{nounibaredI}\textbackslash hline\color{black} \\
1 \color{nounibaredI}\&  \color{black}Nürnberg \color{nounibaredI}\&  \color{black}Wolf \color{nounibaredI}\&  \color{black}Lamm \color{nounibaredI}\&  \color{black}23:10 \color{nounibaredI}\color{nounibaredI}\textbackslash \color{nounibaredI}\textbackslash \color{black} \\
\color{nounibaredI}\color{nounibaredI}\textbackslash hline\color{black} \\
2 \color{nounibaredI}\&  \color{black}Bamberg \color{nounibaredI}\&  \color{black}Meyer \color{nounibaredI}\&  \color{black}Beyer \color{nounibaredI}\color{nounibaredI}\textbackslash \color{nounibaredI}\textbackslash \color{black} \\
\color{nounibaredI}\color{nounibaredI}\textbackslash hline\color{black} \\
3 \color{nounibaredI}\&  \color{black}Zirndorf \color{nounibaredI}\&  \color{black}Brandst. \color{nounibaredI}\&  \color{black}Brauer \color{nounibaredI}\&  \color{black}21:21\color{nounibaredI}\color{nounibaredI}\textbackslash \color{nounibaredI}\textbackslash \color{black} \\
\color{nounibaredI}\color{nounibaredI}\textbackslash hline\color{black} \\
\color{nounibaredI}\color{unibablueI}\textbackslash\color{unibablueI}end\color{black}\color{black}\{tabular\} \\

\end{ttfamily}
\end{footnotesize}

\begin{tabular}{c|p{40mm}|lr|c}
\multicolumn{5}{c}{E-Sports Championship Franconia}
 \\
\hline
\hline
Number & Place & Player 1 & Player 2 & Result \\
\hline
1 & N\"urnberg & Wolf & Lamm & 23:10 \\
\hline
2 & Bamberg & Meyer & Beyer & \\
\hline
3 & Zirndorf & Brandst. & Brauer & 21:21 \\
\hline
\end{tabular}
\end{frame}

\begin{frame}{Tables}
\framesubtitle{Longtable -- Table With Line Break}
\bigskip
„tabular“ shows the table on one page. If it does not fit on the page, the remainder is cut off.\\
For tables longer than one page, a table is needed that performs a division of the table.\\
\textbf{Solution: {\ttfamily longtable}}\\
{\ttfamily longtable} allows a line break in the table. Moreover, {\ttfamily longtable} is an environment, so the  {\ttfamily table}-environment is not needed anymore!\\[3mm]

\begin{ttfamily}
\color{unibablueI}\textbackslash begin\color{black}\{longtable\}\{\textit{definition of columns}\}\\
\textit{table content}\\
\color{nounibaredI}\textbackslash caption\color{black}\{caption\}\\
\color{nounibaredI}\textbackslash label\color{black}\{tab:bsptab2\}\\
\color{unibablueI}\textbackslash end\color{black}\{longtable\}
\end{ttfamily}
\end{frame}


\begin{frame}
\frametitle{Indentations With „tabbing“}
\begin{block}{Control}
\begin{itemize}
\item[\begin{ttfamily}\color{nounibaredI}\textbackslash =\end{ttfamily}]\color{black}set a tab position 
\item[\begin{ttfamily}\color{nounibaredI}\textbackslash $>$\end{ttfamily}]select a tab position
\end{itemize}
\end{block}

\begin{columns}
\begin{column}{.5\textwidth}
\begin{ttfamily}{\scriptsize
\color{nounibaredI}\color{nounibaredI}\textbackslash documentclass\color{black}\{article\} \\
\color{nounibaredI}\color{unibablueI}\textbackslash\color{unibablueI}begin\color{black}\color{black}\{document\} \\
\color{nounibaredI}\color{unibablueI}\textbackslash\color{unibablueI}begin\color{black}\color{black}\{tabbing\} \\
Employ\color{nounibaredI}\color{nounibaredI}\textbackslash \color{black}=ee:\color{nounibaredI}\color{nounibaredI}\textbackslash \color{nounibaredI}\textbackslash \color{black} \\
A  \color{nounibaredI}\color{nounibaredI}\textbackslash \color{black}> Daniel\color{nounibaredI}\color{nounibaredI}\textbackslash \color{nounibaredI}\textbackslash \color{black} \\
B  \color{nounibaredI}\color{nounibaredI}\textbackslash \color{black}> Martin\color{nounibaredI}\color{nounibaredI}\textbackslash \color{nounibaredI}\textbackslash \color{black} \\
C  \color{nounibaredI}\color{nounibaredI}\textbackslash \color{black}> Linus\color{nounibaredI}\color{nounibaredI}\textbackslash \color{nounibaredI}\textbackslash \color{black} \\
xxx\color{nounibaredI}\color{nounibaredI}\textbackslash \color{black}=xxx\color{nounibaredI}\color{nounibaredI}\textbackslash \color{black}=xxxxxxx\color{nounibaredI}\color{nounibaredI}\textbackslash kill\color{black} \\
\color{nounibaredI}\color{nounibaredI}\textbackslash \color{black}> Committees\color{nounibaredI}\color{nounibaredI}\textbackslash \color{nounibaredI}\textbackslash \color{black} \\
\color{nounibaredI}\color{nounibaredI}\textbackslash \color{black}>\color{nounibaredI}\color{nounibaredI}\textbackslash \color{black}> Tests\color{nounibaredI}\color{nounibaredI}\textbackslash \color{nounibaredI}\textbackslash \color{black} \\
\color{nounibaredI}\color{nounibaredI}\textbackslash \color{black}>\color{nounibaredI}\color{nounibaredI}\textbackslash \color{black}> Mails\color{nounibaredI}\color{nounibaredI}\textbackslash \color{nounibaredI}\textbackslash \color{black} \\
\color{nounibaredI}\color{unibablueI}\textbackslash\color{unibablueI}end\color{black}\color{black}\{tabbing\} \\
\color{nounibaredI}\color{unibablueI}\textbackslash\color{unibablueI}end\color{black}\color{black}\{document\} \\
}
\end{ttfamily}
\end{column}
\begin{column}{.5\textwidth}
\begin{tabbing}
Employ\=ee:\\
A  \> Daniel\\
B  \> Martin\\
C  \> Linus\\
xxx\=xxx\=xxxxxxx\kill
\> Committees\\
\>\> Tests\\
\>\> Mails\\
\end{tabbing}
\end{column}
\end{columns}

If you use the command \begin{ttfamily}\color{nounibaredI}\textbackslash kill\color{black}\end{ttfamily}, the rest of the line is not shown. In this way you can do the formatting without showing the respective text.
\end{frame}
