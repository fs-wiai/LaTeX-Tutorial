\color{nounibaredI}\color{nounibaredI}\textbackslash documentclass\color{black}\color{nounibagreenI}[a4paper, pdftex, 12pt, ngerman]\color{black}\{article\} \\
\color{nounibaredI}\color{nounibaredI}\textbackslash usepackage\color{black}\color{nounibagreenI}[utf8]\color{black}\{inputenc\} \\
\color{nounibaredI}\color{nounibaredI}\textbackslash usepackage\color{black}\color{nounibagreenI}[T1]\color{black}\{fontenc\} \\
\color{nounibaredI}\color{nounibaredI}\textbackslash usepackage\color{black}\{babel\} \\
\color{nounibaredI}\color{unibablueI}\textbackslash\color{unibablueI}begin\color{black}\color{black}\{document\} \\
Die Fußnote\color{nounibaredI}\color{nounibaredI}\textbackslash footnote\color{black}\{hier folgt der Fußnotentext\} zu einem Wort oder Text erscheint immer auf der Seite, wo sie hingehört. Der Fußnotentext steht dabei in Klammern.\color{nounibaredI}\color{nounibaredI}\textbackslash \color{nounibaredI}\textbackslash \color{black} \\
\color{nounibaredI}\color{nounibaredI}\textbackslash \color{nounibaredI}\textbackslash \color{black} \\
Eine manuelle Nummerierung ist ebenfalls möglich.\color{nounibaredI}\color{nounibaredI}\textbackslash footnote\color{black}\color{nounibagreenI}[10]\color{black}\{genau so\}, auch ohne Fußnotentext\color{nounibaredI}\color{nounibaredI}\textbackslash footnotemark\color{black}\color{nounibagreenI}[2]\color{black}. \\
\color{nounibaredI}\color{unibablueI}\textbackslash\color{unibablueI}end\color{black}\color{black}\{document\} \\
