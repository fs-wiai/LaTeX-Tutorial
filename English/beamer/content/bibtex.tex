\section{\BibTeX}
\begin{frame}
\frametitle{\BibTeX}
\framesubtitle{Add On f"ur \LaTeX}
\begin{exampleblock}{Neue Pakete in diesem Abschnitt}
\begin{itemize}
\item natbib
\end{itemize}
\end{exampleblock}

\begin{block}{Neue Befehle in diesem Abschnitt}
\begin{itemize}
%\item \color{nounibaredI}\textbackslash renewcommand\color{black}\{\color{nounibaredI}\textbackslash refname\color{black}\}\{...\}
\item \color{nounibaredI}\textbackslash cite\color{black}\{Author2014\}
\item \color{nounibaredI}\textbackslash bibliographystyle\color{black}\{alpha$\mid$abbrv$\mid$natdin$\mid$apa$\mid$etc.\}
\item \color{nounibaredI}\textbackslash bibliography\color{black}\{literature.bib\}
%\item \color{nounibaredI}\textbackslash addcontentsline\color{black}\{toc\}\{section\}\{\color{nounibaredI}\textbackslash refname\color{black}\}
\end{itemize}
\end{block}
\end{frame}

\begin{frame}
\frametitle{\BibTeX und \LaTeX ~in Kombination}
\textbf{Vorteile:}
\begin{itemize}
\item Literaturverzeichnis wird in einer vom Dokument unabh"angigen \texttt{.bib}-Datei gespeichert.
\item Speicherung der Daten im \BibTeX -Format. Hierbei wird nach Quellenart unterscheiden, z.B. mit \color{nounibaredI}$@$book\color{black},~\color{nounibaredI}$@$article\color{black}~usw.
\item Gro\ss e Auswahl an Zitierstilen
\item \textbf{Automatische, dem Style entsprechende, Generierung des Literaturverzeichnises (LVZ)}
\item Aufnahme der Eintr"age in das LVZ nur wenn die Quelle zuvor im Text zitiert wurde
\end{itemize}
\textbf{Nachteile:}
\begin{itemize}
\item Das Erstellen des LVZ mit besonderen Anforderungen ist zum Teil nur erschwert m"oglich
\end{itemize}
\end{frame}

\begin{frame}
\frametitle{\BibTeX und \LaTeX ~in Kombination}
\framesubtitle{\BibTeX -Dateien}
\begin{columns}
\begin{column}{0.52\textwidth}
\textbf{Beispieleintrag:}\\[1em]

\color{nounibaredI}$@$book\color{black}\{Culik93,\\
\color{nounibaredI}title\color{black} = \{Die Welt der Pinguine\},\\
\color{nounibaredI}author\color{black} = \{B.M. Culik and R. P. Wilson\},\\
\color{nounibaredI}publisher\color{black} = \{\{BLV\}M"unchen\},\\
\color{nounibaredI}year\color{black} = \{1993\}\\
\}
\end{column}
\begin{column}{0.55\textwidth}
\textbf{Erkl"arungen zum Eintrag:}
\begin{itemize}
\item \color{nounibaredI}$@$book\color{black}~- Angabe der Quellenart, hier also ein Buch
\item Culik93 - Definition eines eindeutigen Referenzierungsschl"ussels
\item \color{nounibaredI}author\color{black}~- Autor des Buches
\item \color{nounibaredI}title\color{black}~- Titel des Buches
\item \color{nounibaredI}publisher\color{black}~- Verlag
\item \color{nounibaredI}year\color{black}~- Erscheinungsjahr
\end{itemize}
\end{column}
\end{columns}
\begin{alertblock}{Achtung: Reihenfolge beim Kompilieren beachten!}
(1) pdflatex (2) bibtex (3) pdflatex (4) pdflatex
\end{alertblock}
\end{frame}

\begin{frame}
\frametitle{\BibTeX und \LaTeX ~in Kombination}
\framesubtitle{\"Ubersicht "uber die Befehle}
\begin{itemize}

\item \color{nounibaredI}\textbackslash cite\color{black}\{Culik93\} \hfill Zitieren eines \BibTeX Eintrages\\

\item \color{nounibaredI}\textbackslash bibliographystyle\color{black}\{alphadin\} \hfill Auswahl des LVZ-Stils \glqq \texttt{alphadin}\grqq\\

\item \color{nounibaredI}\textbackslash bibliography\color{black}\{bibliography.bib\} \hfill Angabe der  \BibTeX-Datei\\

\item \color{nounibaredI}\textbackslash usepackage\color{black}\{natbib\} \hfill F"ur den Zitierstil \texttt{natdin} notwendig\\
%\item \color{nounibaredI}\textbackslash renewcommand\color{black}\{\color{nounibaredI}\textbackslash refname\color{black}\}\newline \{Literaturverzeichnis\} \hfill Anpassung des Namens von Standard Literatur auf Literaturverzeichnis

%\item \color{nounibaredI}\textbackslash addcontentsline\color{black}\{toc\}\{section\}\newline \{\color{nounibaredI}\textbackslash refname\color{black}\} \hfill Aufnahme des LVZ ins Inhaltsverzeichnis

\end{itemize}


\end{frame}


\begin{frame}
\frametitle{\BibTeX und \LaTeX ~in Kombination}
\framesubtitle{Beispiele für Styles}
\image{\textwidth}{image/alpha.png}{Alpha Zitierstil}{img:alpha}
\end{frame}


\begin{frame}
\frametitle{\BibTeX und \LaTeX ~in Kombination}
\framesubtitle{Beispiele für Styles cont'd}
\image{\textwidth}{image/IEEEtran.png}{IEEEtran Zitierstil}{img:ieeetran}
\end{frame}

\begin{frame}
\frametitle{\BibTeX und \LaTeX ~in Kombination}
\framesubtitle{Beispiele für Styles cont'd}
\image{\textwidth}{image/natdin.png}{Natdin Zitierstil}{img:natdin}
\end{frame}


\begin{frame}
\frametitle{\BibTeX und \LaTeX ~in Kombination}
\framesubtitle{Beispiele für Styles cont'd}
\image{\textwidth}{image/apa.png}{Apa Zitierstil}{img:apa}
\end{frame}