
\begin{frame}
\frametitle{Referenzen}
\framesubtitle{Abbildungen einfügen – A closer look}
\begin{tabbing}
\begin{ttfamily}\color{nounibaredI}\textbackslash label\color{black}\{Labelname\}\end{ttfamily} \=Mit diesem Befehl
setzt man ein Label. Später im\\\> Text kann man dann durch eine Referenz auf dieses\\
\> Label verweisen.\\
\> Dies geschieht mit dem Befehl
\begin{ttfamily}\color{nounibaredI}\textbackslash ref\color{black}\{Labelname\}\end{ttfamily}.
\end{tabbing}
\begin{ttfamily}Der kleine Tux ist ein Allesfresser. Egal ob Gem"use oder
Schnittlauch, nichts ist vor ihm sicher. (siehe Bild
\color{nounibaredI}\textbackslash ref\color{black}\{img:tux1\})\end{ttfamily}\\[3mm]
\textbf{Ergebnis:}\\[3mm]
\begin{minipage}{\textwidth}\begin{rm}
Der kleine Tux ist ein Allesfresser. Egal ob Gem"use oder
Schnittlauch, nichts ist vor ihm sicher. (siehe Bild
1)\end{rm} \end{minipage}\\[3mm]
\textbf{Und warum das Ganze?}\\
Durch solche Referenzen wird immer auf das richtige Bild verwiesen, auch wenn zwischendurch noch weitere Bilder einfügt wurden.
\end{frame}
