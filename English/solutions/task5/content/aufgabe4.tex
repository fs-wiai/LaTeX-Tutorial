\section{Aufgabe 4}
\subsection{Tabellen \& Formeln}
Die Tabelle \ref{tab:ani} besteht aus 3 Spalten:\\
Die erste Spalte ist mit einem p von 25mm definiert. Die zweite und die dritte Spalte sind zentriert.

\begin{longtable}{p{25mm}|c|c}
& Fuchs & Elster\\
\hline
lat. Bezeichnung & Vulpes & Pica\\
\hline
Familie & Hunde & Rabenvögel\\
\hline
Ordnung & Raubtier & Sperrlingsvogel\\
\hline
Gewicht & m: 6,6kg w: 5,5kg & 200--250g\\
\hline
Länge  & m: 71,4cm w: 67,8cm & 46cm\\
\hline
Geschwindigkeit $ = \sqrt{v\cdot v}$ & $55\frac{km}{h}$ & mind. superschnell \\
\hline
Farbe & rötlich & schwarz\\
\hline
\caption{Wild Animals}
\label{tab:ani}
\end{longtable}

\subsection{Aufzählungen}
Um bei den vielen Verschachtelungen nicht den Überblick zu verlieren, sind Einrückungen der items sinnvoll.
\begin{enumerate}
  \item 
  \begin{enumerate}
    \item Vorteile des Fuchses:
    \item
    \begin{itemize}
      \item schlau
      \item schaut cool aus
    \end{itemize}
    \item Nachteile des Fuchses:
    \item
    \begin{itemize}
      \item Pelz wird verarbeitet
      \item sehr viele Autos fahren gerne über Füchse
    \end{itemize}
    \item weit verbreitet aber gefährdet
  \end{enumerate}
  \item
  \begin{enumerate}
    \item Vorteile der Elster:
    \begin{itemize}
      \item kann fliegen
      \item muss nicht umziehen
    \end{itemize}
    \item Nachteile der Elster:
    \begin{itemize}
      \item Diebischkeit wird bestraft
      \item viele landen hinter Gittern
    \end{itemize}
      \item singt ganz gut, aber ist gefährlich
  \end{enumerate}
\end{enumerate}

\subsection{Tabbing}

Hier wurde eine Zeile  um die Ecke gebracht.\\

\begin{tabbing}

Allerletzte \=xxxxx \= xxxxx \= xxxxx \= xxx \= xxxxx \= xxxxx \= xxxxx \= xxxxx\= xxxxx\kill
Spitze\>\>\>\>\>Süßwaren\\
2.Reihe\>\>\>\>Milch\>\>Eier\\
3.Reihe\>\>\>Nüsse \>\>\>\>Samen\\
Vorletzte\>\>Obst \>\>\>\>\>\>Gemüse\\
Allerletze \>Vollkornprodukte\>\>\>\>\>\>\> \>Hülsenfrüchte\\
\end{tabbing}

